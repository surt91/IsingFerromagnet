\subsection{Thermodynamic Theory}
\label{ssec:theory}
    %~ To examine the disordered Ising model in Equilibrium,
    %~ Its temperature \(T\),
    %~ volume and number of particles \(N\) are constant, so that it is a canonical
    %~ ensemble.
    In a canonical ensemble the probability \(p_i\) of a state
    \(i\) is distributed according to a Boltzman distribution
    \[p_i \propto e^{-\frac{\hat{H}}{k_B T}}\]
    Further the free energy \(F\) of an canonical
    ensemble is minimized and all observables can be derived from \(F\)
    in a straight forward way, as stated in every textbook about
    statistical physics (e.g. \cite{nolting2005}).\\
    Because of \(F=U-TS\) where \(U=\avg{H}\) is the internal energy and
    \(S\) entropy, one can guess, that for low \(T\) the internal energy
    will be low, and for big \(T\) the entropy will be high, to minimize
    \(F\). The Hamiltonian of the Ising model creates an conflict between
    \(U\) and \(S\) by assigning low energies to states with high order.
    So \(U\) is low if all spins are aligned, where \(S\) as a measure of
    disorder is low. This preliminary considerations make a phase
    transition at some \(T\), where a high entropy lowers \(F\) more than
    a low internal energy, very plausible.\\\
    To determine
    \[F=-k_{B}T \ln{Z}\]
    one has to know the partition function
    \[Z=\sum_i e^{-\frac{\hat{H}_i}{k_{B}T}}\]
    where the sum goes over all possible states \(i\) of the system.
    For small \(N\) the partition function is computable, but the system
    may show very different properties than in the thermodynamic limit.
    To minimize these finite size effects, it is desirable to examine
    systems with large \(N\). But then there are of course next to
    infinity many different states, such that it is unfeasible to
    calculate the energy for each state, except for cases, where it is
    possible to solve it analytically.\\
    If not, one can get estimates of the observables for big \(N\) using
    Monte Carlo simulations.\\

\subsection{Monte Carlo Simulations}
\label{ssec:montecarlo}
    The idea behind Monte Carlo simulations is to take random samples of
    the observable, which should be measured, and estimate the observable from
    this samples. In example the Monte Carlo integration chooses random
    points within the bounding box of the function and estimates the integral
    as the fraction of points below the function and points above multiplied
    by the area of the bounding box.
    Of course one wants to sample more points, where the area below the
    function is big, because points in areas, where the function does
    contribute little to the integral, contribute little to the integral.
    This can be achieved by \emph{Importance Sampling}. One
    generates random numbers according to some probability distribution, which
    yields more numbers in the range where the function to integrate has high
    values and weights the result accordingly.\\
    The same principle is used to sample properties of statistical
    systems by generating random states.
    As noted before the system under scrutiny is canonical and therefore
    the states are distributed according to the Boltzman distribution.
    Hence importance sampling can be utilized.\\
    But it is difficult to create a random state of a physical system e.g. the
    Ising system in equilibrium, hence one uses \emph{Markov Chains} to
    generate new states \(\nu\) from old ones \(\mu\).
    It is important that the transition probabilities \(A(\mu \to \nu)\)
    obey \emph{Detailed Balance} and \emph{Ergodicity}.
    \emph{Detailed Balance} means that the probability to leave a state is
    the same as the probability to enter the state in equilibrium
    \(p_\mu A(\mu \to \nu) = p_\nu A(\nu \to \mu)\) with \(p_\mu\) the
    probability to be in state \(\mu\).
    And \emph{Ergodicity} requires that every possible state is reachable
    from every other state in finite time. \cite{NewmanBarkema1999} \cite{Katzgraber2011}
    Otherwise the samples might not be representative for the whole system.\\
    In this thesis three algorithms, which fulfill all requirements,
    were used. They will be shortly described in the following subsections.

    \subsubsection{Metropolis}
        A Metropolis Monte Carlo\cite{Metropolis1953} simulation of an
        Ising model will choose a random spin, calculate the energy change
        \(\Delta \hat H\) defined in eq. \eqref{eq:dH} that would result
        from a flip of that spin and execute the flip with the probability \(A\)
        eq. \eqref{eq:metropolis} \cite{NewmanBarkema1999} \cite{Katzgraber2011}.
        \begin{equation}
            \Delta \hat H = \hat H(\nu) - \hat H(\mu)\\
            \label{eq:dH}
        \end{equation}
        \begin{equation}
            A(\mu \to \nu)
            \begin{cases}
                1                                 & \Delta \hat H \le 0 \\
                \exp{\brac{-\beta \Delta \hat H}} & \Delta \hat H > 0
            \end{cases}
            \label{eq:metropolis}
        \end{equation}
        So if a transition lowers the energy it is always done. This
        results in a high ratio between chosen spins and flipped spins.
        Therefore it minimizes the calculations needed for a change of
        the state.

    \subsubsection{Wolff}
    \label{sssec:wolff}
        Close to the critical temperature \(T_c\) the Metropolis
        gets slower. This is called \emph{critical slowing} down and the
        cause is beyond the scope of this thesis.\\
        Using a cluster algorithm like the Wolff
        algorithm \cite{Wolff1989} speeds things up.
        For an Ising model the Wolff algorithm builds a cluster of sites
        starting with a random site and adding neighboring sites of the
        same spin with probability \(P_{\mathrm{add}}\) from eq. \eqref{eq:wolffAdd}
        \begin{equation}
            P_{\mathrm{add}} = 1-\exp\brac{-2\beta J}
            \label{eq:wolffAdd}
        \end{equation}
        Where \(J\) is the coupling constant (explained in section
        \ref{ssec:isingmodel}). The neighboring sites of the added sites
        are also considered and so forth. When there are no more sites
        to add, the spin of every site in the cluster is flipped
        \cite[S. ??]{NewmanBarkema1999} \cite[S. 151f]{Katzgraber2011}.
        This leads fast to new uncorrelated states at the critical
        temperature because big clusters are flipped. But there are not
        much advantages at high or low temperatures. At low temperatures
        the cluster will consist of almost all sites such that all but
        very few spins will be flipped. At high temperatures the cluster
        will only contain very few sites.
        Both situations have no advantage against the Metropolis algorithm.\\
        So one would activate this algorithm near the critical temperature
        but use a simple Metropolis algorithm at high and low temperatures.

    \subsubsection{Parallel Tempering}
        %~ The main aim is to obtain the critical temperatures
        %~ \(T_c\) for different disorder paramters \(\sigma\).
        %~ Therefore it is necessary to simulate for many temperatures,
        %~ so that \emph{Parallel Tempering}\footnote{Before R. H.
            %~ Swendsen published this paper, a algorithm \(MC^3\) was
            %~ already published with the same idea. [citation needed]}
        %~ \cite{ParallelTempering1986} is a suited algorithm.
        Parallel Tempering\cite{ParallelTempering1986} simulates many identical systems at different
        temperatures and periodically swaps the spin configurations
        between two neighboring temperatures with probability \(P\) from
        eq. \eqref{eq:partemp} \cite[S. ??]{NewmanBarkema1999} \cite[S. 155ff]{Katzgraber2011}.
        \begin{equation}
            P((E_i,T_i) \to (E_{i+1},T_{i+1})) = \min\brac{1,\exp\brac{\brac{E_{i+1}-E_i}\brac{\frac{1}{T_{i+1}}-\frac{1}{T_i}}}}
            \label{eq:partemp}
        \end{equation}
        This has the advantages that correlation times of single
        temperatures are far smaller because their spin configuration
        gets often replaced by an other uncorrelated configuration. In
        many cases the more important advantage is, that a system which
        is trapped in a local minimum at a given temperature, can travel
        to higher temperatures, leave its local minimum and cool down
        again in a lower minimum.\\
        In the case of a ferromagnetic Ising model the risk to get trapped
        in an local energy minimum is very low. Though the autocorrelation
        decreases significantly. In the scope of this thesis this comes
        for no cost, because one has to simulate for many temperatures
        to determine the critical temperature. The additional calculations
        to determine weather to swap configurations or not are small in
        comparison with those that would be needed to generate a new
        uncorrelated state without \emph{parallel tempering}.

    \subsubsection{Implementation Details}
        Here a mixture of the above three algorithms is used.
        Each sweep \(N\) Metropolis spin flips, one Wolff cluster flip
        and one parallel tempering swap are performed. Where \(N\) is the
        count of sites.\\
        Because it is not known before, where the critical temperatures
        \(T_c\) are located, the Wolff cluster algorithm is used for
        every temperature. The speed up at criticality is worth the
        moderate slow down at other temperatures.

    \subsubsection{Equilibration- and Autocorrelation Time}
    \label{sssec:eqtime}
        To generate an equilibrium state one starts with an arbitrary state
        and waits until it is equilibrated. The count of sweeps till
        equilibrium is called \emph{equilibration time} \(t_{eq}\).
        All measurement should start after this time.\\
        To determine when the system is in equilibrium, one can watch the
        development of some observables and take the point at which there
        are no big changes anymore as equilibrium. In fig.
        \ref{fig:equiandauto}\subref{sfig:equiandauto:equiE}
        the equilibrium is reached after approximately \(N=50\) sweeps for
        either an initial condition of all spins up and all spins random. It
        does not harm to double that value to be save.
        \begin{figure}[htbp]
            \centering
            \subfigure[Example of an Equilibrating Ising System][]{
                    \label{sfig:equiandauto:equiE}
                    \includegraphics[width=0.47\textwidth]{plots/equiE}
            }
            \subfigure[Example of the Autocorrelation of an Ising System][]{
                    \label{sfig:equiandauto:autoM}
                    \includegraphics[width=0.47\textwidth]{plots/autoM}
            }
            \caption[Examples for Equilibration and Autocorrelation]
            {
                \subref{sfig:equiandauto:equiE} Example of a Ising system
                    \(L=64\) reaching equilibrium at \(T=2.36\) and
                \subref{sfig:equiandauto:autoM} the autocorrelation of an
                    Ising system \(L=64\) at \(T=2.40\) (only Metropolis
                    sweeps -- otherwise the decline is too steep to show)
                    on half logarithmic axis.
                    The straight line is an exponential fit \(\exp(-t/\tau)\)
                    with \(\tau = 342(1)\).
            }
            \label{fig:equiandauto}
        \end{figure}\\
        Because every state is generated from the state before, measurements
        of subsequent states are correlated. To determine when two states
        are independent, one calculates the normalized autocorrelation function
        \(\frac{\chi(t)}{\chi(0)}\), which should decay exponentially
        \(\chi(t) \propto \exp(t/\tau)\). This is visible in the half
        logarithmic plot \ref{fig:equiandauto}\subref{sfig:equiandauto:autoM}.
        To get the autocorrelation time one can either fit a exponential
        function \(\exp(-t/\tau)\) like in fig \ref{fig:equiandauto}\subref{sfig:equiandauto:autoM}
        or integrate \(\tau = \int \frac{\chi(t)}{\chi(0)} \de t\).
        \(\tau\) is an estimate after which time two samples are not
        correlated anymore. \cite[S. ??]{NewmanBarkema1999} \cite[S. 150f]{Katzgraber2011}.
        To make sure that the error is not underestimated one should wait
        \(2\tau\) sweeps between two measurements.
        The autocorrelation time is of course dependent on the temperature.
        For example for the standard Metropolis algorithm the fluctuations
        are strong at high temperatures and subsequent
        states are more dissimilar and therefore less correlated than at low
        temperatures, where less spins are flipping. But the longest
        autocorrelation times are encountered at the critical temperature.
        This effect is called \emph{critical slowing down} and is
        characterized by the \emph{dynamical critical exponent} \(z\)
        \cite{SwendsenWang1987}. The dependence of the autocorrelation time
        \(\tau\) on the system size \(L\) is then given by \(\tau \propto L^z\).
        As mentioned in section \ref{sssec:wolff}, the Wolff cluster algorithm
        decreases \(z\) dramatically. This causes the course of the autocorrelation
        time in dependence on temperature to change significantly as shown in
        section \ref{sec:results}.
