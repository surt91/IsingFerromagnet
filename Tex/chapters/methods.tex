\subsection{Thermodynamic Theory}
\label{ssec:theory}
    The disordered Ising model will be examined as a \emph{canonical system} in
    \emph{equilibrium}. A canonical system can exchange energy with a
    heat bath, thus it has a constant temperature equal to the temperature of the
    heat bath.
    Equilibrium is defined as a steady state, wherein
    the observables are only fluctuating but not changing in any
    particular direction. For example thermal equilibrium denotes the
    condition that the system under scrutiny has reached the temperature
    of the heat bath. This is the case once no net energy exchange occurs,
    thus the energy of the simulated system reaches a steady state, where
    only fluctuations occur.
    In a canonical ensemble the probability \(p_\nu\) of a state
    \(\nu\) is distributed according to a Boltzmann distribution
    \begin{equation}
        p_\nu = \frac{1}{Z} e^{-\beta H_\nu}
    \end{equation}
    \begin{equation}
        \beta = \frac{1}{k_B T}
    \end{equation}
    Where \(Z\) is the partition function, which normalizes \(p_\nu\) to
    a probability: \(\sum_\nu p_\nu = 1\).
    Further the free energy \(F\) of a canonical ensemble is minimized
    in equilibrium and all observables can be derived from \(F\)
    in a straight forward way, as stated in every textbook about
    statistical physics (e.g.\ Ref.\ \cite{nolting2005}).\\
    Because of
    \begin{equation}
        F = U - TS
    \end{equation}
    where \(S\) is the entropy, \(U=\avg{H}\) the internal energy and
    \(\avg{\cdot}\) declares the expectation value of an observable.
    One can guess that for low \(T\) the internal energy
    will be low, and for high \(T\) the entropy will be high to minimize
    \(F\).
    Considering the Hamiltonian of the Ising model, a spin configuration
    of high order, where most spins are aligned with their neighbors,
    leads to a low value of \(H\) and therefore a low value of \(U\).
    Simultaneously, this state of high order corresponds to a low entropy \(S\).
    Analogically a state of high \(U\) is also a state of high \(S\).
    These preliminary considerations make a phase transition at some
    \(T\) where the influence of the entropy on \(F\) becomes the same
    order of magnitude as the influence of the internal energy on \(F\),
    very plausible.\\
    To determine
    \begin{equation}
        F=-k_{B}T \ln{Z}
    \end{equation}
    one has to know the partition function
    \begin{equation}
        Z = \sum_\nu e^{-\beta H_\nu},
        \label{eq:partitionFunction}
    \end{equation}
    where the sum goes over all possible states \(\nu\) of the system.
    Then averages can be computed according to
    \begin{equation}
        \avg{O} = \frac{1}{Z} \sum_\nu O_\nu e^{-\beta H_\nu}.
    \end{equation}
    Because every site can have two states, there are \(2^N\) different
    states of the system. For each the energy \(H_\nu\) has to be calculated
    to solve the sum from Eq.\ \eqref{eq:partitionFunction} to gain \(Z\).
    Hence for small \(N\) the partition function is computable, but the system
    may show very different properties than in the thermodynamic limit.
    To minimize these finite size effects it is desirable to examine
    systems with large \(N\). But \(2^N\) is a very rapidly increasing
    number, so for large \(N\) it is unfeasible to calculate the energy for each state, except
    for cases where it is possible to solve it analytically.\\
    If not, one can get estimates of the observables for big \(N\) using
    Monte Carlo simulations, which are introduced in the next chapter.\\

    The observables which are measured in this thesis are the
    magnetization per spin
    \begin{equation}
        m = \frac{1}{N} \sum_i s_i
    \end{equation}
    and the energy per spin
    \begin{equation}
        E = \frac{1}{N} H.
    \end{equation}
    As mentioned before, properties near the phase transition and the
    critical temperature \(T_c\), where the phase transition occurs, will
    be examined. The Ising system in two dimensions shows a second order
    phase transition, hence \(m\) and \(E\) are continous, but show at
    \(T_c\) an infinitly sharp slope in the thermodynamic limit, i.e.\
    the first derivative diverges.
    From statistical physics (See Ref.\ \cite{nolting2005}) it is known that
    these derivatives can be expressed by fluctuations, e.g.\ the specific
    heat can be expressed as
    \begin{equation}
        c = \frac{\partial \avg{H}}{\partial T} = k_B \beta^2 \avg{\brac{H-\avg{H}}^2}.
    \end{equation}
    The specific heat is a measure for how much energy is needed to change
    the temperature of the system.
    Analogically the susceptibility
    \begin{equation}
        \chi = N \beta \avg{\brac{m-\avg{m}}^2}
    \end{equation}
    is a measure for how strong an outer magnetic field changes the
    magnetization of the system.
    Beside these observables from classical physics, a fifth observable
    the Binder cumulant \cite{Binder1981}
    \begin{equation}
        g = \frac{3}{2}\brac{1-\frac{\avg{m^4}}{3\avg{m^2}^2}}
        \label{eq:binder}
    \end{equation}
    is considered.
    This is a dimensionless value, which can be used to determine the
    critical point. These five observables will be sufficient to analyse
    the phase transition in the scope of this bachelor thesis. All of them
    can be easily computed when \(m\) and \(E\) are measured. The next
    chapter will show, how to get estimates for them through Monte Carlo
    simulations.

\subsection{Monte Carlo Simulations}
\label{ssec:montecarlo}
    The idea behind Monte Carlo simulations is to take random samples of
    the observable, which should be measured, and to estimate the mean of
    the observable from this samples. To apply this technique to statistical
    ensembles, one creates sample states of the system, measures the
    observables and calculates the expected value through averaging.\\
    In statistical physics the expected value of an observable \(O\)
    is -- as also noted above -- calculated by
    \begin{equation}
        \avg{O} = \frac{1}{Z} \sum_\nu p_\nu O_\nu.
        \label{eq:estim}
    \end{equation}
    It is however possible that there are few states contributing
    massively more than others. In canonical systems at low \(T\)
    states with low values of \(H\) contribute much more than states with
    high values of \(H\).
    But if one samples the \(2^N\) states evenly, it is probable to miss
    them. This is called \emph{simple sampling} and results in large
    errorbars for any observable.
    It is therefore desirable to sample only the states with high
    contributions to the sum.
    As noted before, the system under scrutiny is canonical and therefore
    \(p_\nu\) is known. The states are distributed according to the Boltzmann
    distribution. Hence \emph{Importance Sampling} can be utilized.\\
    Instead of sampling uniformly distributed random states, one should sample
    states according to their occurence probability given by the Boltzmann
    distribution. In fact this reduces the estimator Eq.\ \eqref{eq:estim} to
    \begin{equation}
        O_M = \frac{1}{M} \sum_{\nu=1}^M O_\nu,
    \end{equation}
    where \(M\) is the number of samples. The proof is shown in \cite{NewmanBarkema1999}.
    This is a very convenient form.\\
    But it is difficult to create a random state of a physical system,
    e.g.\ the Ising system, according to a given distribution. The simple
    approach of creating uniformly distributed random states and reject
    them with probability \(p_\nu^{-1}\) depending on their energy is not
    efficient, because many generated states will be discarded and the
    computing time to generate them and calculating their energy will
    be wasted.
    Hence one uses \emph{Markov Chains} to generate new states \(\nu\)
    from former ones \(\mu\). It is important that the transition probabilities
    \(A(\mu \to \nu)\) obey \emph{Detailed Balance} and \emph{Ergodicity}.
    \emph{Detailed Balance} means that the probability to leave a state is
    the same as the probability to enter the state in equilibrium
    \(p_\mu A(\mu \to \nu) = p_\nu A(\nu \to \mu)\) with \(p_\mu\) the
    probability to be in state \(\mu\).
    \emph{Ergodicity} requires that every possible state is reachable
    from every other state in finite time, see Refs.\ \cite{NewmanBarkema1999} \cite{Katzgraber2011}.
    This ensures that the system can equilibrate and that the states are
    distributed according to the desired distribution in equilibrium
    \cite{NewmanBarkema1999}. Otherwise the samples might not be representative
    for the whole system.\\
    In this thesis three algorithms, which fulfill all requirements,
    were used. They will be shortly described in the following subsections.
    But first equilibration- and autocorrelation time will be discussed.

    \subsubsection{Equilibration- and Autocorrelation Time}
    \label{sssec:eqtime}
        To generate states acoording to the Boltzmann distribution at a
        given temperature \(T\), one starts with an arbitrary state
        and waits until it reaches thermal \emph{equilibrium}. Because
        equilibrium is defined as a steady state, one can determine it by
        observing the change of the observables over the progressing
        simulation as pictured in Fig.\ \ref{fig:equiandauto}\subref{sfig:equiandauto:equiE}.
        The progression of a Monte Carlo simulation is measured in \emph{sweeps},
        which denote some operation. The count of sweeps until
        equilibrium is reached, is called \emph{equilibration time} \(t_{eq}\).
        All measurements should start after this time.\\
        In Fig.\ \ref{fig:equiandauto}\subref{sfig:equiandauto:equiE}
        the equilibrium is reached after approximately \(N_{s} \approx 100\) sweeps for
        both an initial condition of all spins up and all spins random. It
        does not harm to double that value to be save. Particularly, because
        it is a random process, so that there can not be an exact value.
        \begin{figure}[htbp]
            \centering
            \subfigure[Example of an Equilibrating Ising System][]{
                    \label{sfig:equiandauto:equiE}
                    \includegraphics[width=0.47\textwidth]{plots/equiE}
            }
            \subfigure[Example of the Autocorrelation of an Ising System][]{
                    \label{sfig:equiandauto:autoM}
                    \includegraphics[width=0.47\textwidth]{plots/autoM}
            }
            \caption[Examples for Equilibration and Autocorrelation]
            {
                \subref{sfig:equiandauto:equiE} Example of an Ising system
                    \(L=64\) reaching thermal equilibrium at \(T=2.36\) after
                    approximately \(N_s=100\) sweeps.\\
                \subref{sfig:equiandauto:autoM} The autocorrelation of an
                    Ising system \(L=64\) at \(T=2.40\) (only Metropolis
                    sweeps -- otherwise the decline is too steep to show)
                    on half logarithmic axis.
                    The straight line is an exponential fit \(\exp(-t/\tau)\)
                    with \(\tau = 342(1)\).
            }
            \label{fig:equiandauto}
        \end{figure}\\
        Because every state is generated from the preceding state, measurements
        of subsequent states are correlated. To determine when two states
        are independent, one calculates the normalized autocorrelation function
        \(\frac{\chi(t)}{\chi(0)}\) with
        \begin{equation}
            \chi(t) = \int \mathrm{d} t' \, [m(t') -\avg{m}][m(t'+t)-\avg{m}],
        \end{equation}
        which is expected to decay exponentially
        \(\chi(t) \propto \exp(t/\tau)\). This is visible in the semilogarithmic
        plot shown in  Fig.\ \ref{fig:equiandauto}\subref{sfig:equiandauto:autoM}.
        To get the autocorrelation time one can either fit an exponential
        function \(\exp(-t/\tau)\) like in Fig.\ \ref{fig:equiandauto}\subref{sfig:equiandauto:autoM}
        or integrate
        \begin{equation}
            \tau = \int \frac{\chi(t)}{\chi(0)} \de t.
            \label{eq:tau}
        \end{equation}
        \(\tau\) is an estimate that specifies the time after which two
        samples are not correlated anymore, see Refs.\ \cite[p. 59ff]{NewmanBarkema1999} \cite[p. 150f]{Katzgraber2011}.
        To ensure that the error is not underestimated, one should wait
        \(2\tau\) sweeps between two measurements.
        The autocorrelation time is dependent on the temperature.
        For example for the standard Metropolis algorithm the fluctuations
        are strong at high temperatures and subsequent
        states are more dissimilar and therefore less correlated than at low
        temperatures, where less spins are flipping. But the longest
        autocorrelation times are encountered at the critical temperature.
        This effect is called \emph{critical slowing down} and is
        characterized by the \emph{dynamical critical exponent} \(z\)
        \cite{SwendsenWang1987}. The dependence of the autocorrelation time
        \(\tau\) on the system size \(L\) is at \(T_{c}\) given by \(\tau \propto L^z\).
        More general, the power law scaling \(\tau \propto \xi^z\) holds, where
        \(\xi\) is the \emph{correlation length}. It diverges at
        \(T_{c}\) and is then limited by the size of the simulated lattice.
        As mentioned in Sec.\ \ref{sssec:wolff}, the Wolff cluster algorithm
        decreases \(z\) dramatically. This causes the course of the autocorrelation
        time in dependence on temperature to change significantly as shown in
        Sec.\ \ref{ssec:results:autocorr}. Also according to Ref.\ \cite{NewmanBarkema1999}
        \(z\) is independent of the lattice structure, which ensures that
        the simulation will benefit from the Wolff cluster algorithm at
        any \(\sigma\).

    \subsubsection{Single Spin Flip Metropolis Update}
        A \emph{Metropolis} Monte Carlo \cite{Metropolis1953} simulation of an
        Ising model will choose a random spin, calculate the energy change
        \begin{equation}
            \Delta H = H_\nu - H_\mu
            \label{eq:dH}
        \end{equation}
        that would result from a flip of that spin. Where \(\mu\) is the
        state before the flip and \(\nu\) after the flip. The flip is then executed
        with the probability
        \begin{equation}
            A(\mu \to \nu) =
            \begin{cases}
                1                            & \Delta H \le 0 \\
                \exp{\brac{-\beta \Delta H}} & \Delta H > 0
            \end{cases}.
            \label{eq:metropolis}
        \end{equation}
        see Refs.\ \cite{NewmanBarkema1999} \cite{Katzgraber2011}.
        So if a transition lowers the energy it will be always done. This
        results in a high ratio between flipped spins and chosen spins.
        Therefore it minimizes the calculations needed for a change of
        the state. Also note that \(\Delta H\) is easy to calculate,
        because it is only affected by the spin of the neighbors of the
        chosen site.

    \subsubsection{Wolff Cluster Update}
    \label{sssec:wolff}
        Close to the critical temperature \(T_c\) the efficiency of the
        single spin flip Metropolis update decreases significantly, i.e.\ the
        autocorrelation time \(\tau\) diverges.
        This is called \emph{critical slowing} down.\\
        In order to circumvent this, a cluster algorithm like the \emph{Wolff}
        algorithm \cite{Wolff1989} can be used.
        For an Ising model the Wolff algorithm builds a cluster of sites
        starting with a random site and adding neighboring sites exhibiting the
        same spin orientation with probability
        \begin{equation}
            P_{\mathrm{add}} = 1-\exp\brac{-2\beta J},
            \label{eq:wolffAdd}
        \end{equation}
        where \(J\) is the coupling constant (c.f.\ Sec.\ \ref{ssec:isingmodel}).
        For every site that is added, the neighboring sites of it are
        also considered for addition. (In the case that they are added,
        they are "added sites" and thus their neighbors get a chance to be
        added too.)
        This procedure continues until there are no more sites to add.
        Then the spin of every site in the cluster is flipped
        \cite[p. 91ff]{NewmanBarkema1999} \cite[p. 151f]{Katzgraber2011}.
        This leads fast to new uncorrelated states at the critical
        temperature because big clusters are flipped. But there are not
        much advantages at high or low temperatures. At high temperatures the cluster
        will only contain very few sites, so only few spins will be flipped.
        At low temperatures the cluster will consist of almost all sites
        such that all but very few spins will be flipped. Through this effect the equilibrium
        will be reached sooner at low temperatures than with the Metropolis
        algorithm, but after equilibration the Wolff cluster update has no
        advantage compared to the Metropolis at high or low temperatures.\\
        So one would activate this algorithm near the critical temperature
        but would use a simple Metropolis algorithm at high and low temperatures.

    \subsubsection{Parallel Tempering}
        In simulations using \emph{parallel tempering} \cite{ParallelTempering1986}
        many identical systems at different temperatures are simulated and
        the spin configurations between two neighboring temperatures are
        swapped periodically with probability \cite[p. 169ff]{NewmanBarkema1999} \cite[p. 155ff]{Katzgraber2011}
        \begin{equation}
            P_{\nu,\nu+1}(S_\nu \leftrightarrow S_{\nu+1}) = \min\brac{1,\exp\brac{\brac{E_{\nu+1}-E_\nu}\brac{\frac{1}{T_{\nu+1}}-\frac{1}{T_\nu}}}},
            \label{eq:partemp}
        \end{equation}
        as schematically pictured in Fig.\ \ref{fig:parTemp}\subref{sfig:parTemp:schema}.
        This has the advantage that correlation times of single
        temperatures are far smaller, because their spin configuration
        often gets replaced by another uncorrelated configuration. In
        many cases the more important advantage is that a system, which
        is trapped in a local minimum at a given temperature, can travel
        to higher temperatures, leave its local minimum and cool down
        again in a lower minimum. If a system is trapped in such a
        metastable state, ergodicity is not guaranteed anymore.
        This is schematically pictured in Fig.\ \ref{fig:parTemp}\subref{sfig:parTemp:E}.
        \begin{figure}[htbp]
            \centering
            \subfigure[Schematic Diagramm of the Algorithm][]{
                \label{sfig:parTemp:schema}
                \includegraphics{images/parTempSchema}
            }
            \subfigure[Schematic Diagramm of the Energy Landscape][]{
                \label{sfig:parTemp:E}
                \includegraphics{images/parTempE}
            }
            \caption[Visualisation of the Parallel Tempering Algorithm]
            {
                \subref{sfig:parTemp:schema} Schematic representation of
                the swapping of spin configurations of different simulations \(S_i\)
                between temperatures.\\
                \subref{sfig:parTemp:E} Sketch of an energy landscape, where
                the state of the system (filled circle) is trapped in an local
                minimum. At low temperatures it is very unlikley that it
                overcomes the energy barrier \(\Delta E\) to the minimum.
                After a swap to higher energies, the barrier can be overcome
                and after a swap to lower energies again, the state in
                the minimum can be reached (open circle).
            }
            \label{fig:parTemp}
        \end{figure}\\
        In the case of a ferromagnetic Ising model the risk to get trapped
        in a local energy minimum is very low. In the scope of this thesis it is benefical
        to use parallel tempering, because one has to simulate at many temperatures
        to determine the critical temperature. The additional calculations
        to determine whether to swap configurations or not, are small in
        comparison with the much lower autocorrelation times which are
        achieved with parallel tempering.

    \subsubsection{Implementation Details}
        Here, a mixture of the above three algorithms is used.
        For each sweep \(N\) single spin flip Metropolis updates, one
        Wolff cluster update and one parallel tempering swap are
        performed.\\
        Because it is not known before, where the critical temperatures
        \(T_c\) are located, the Wolff cluster algorithm is used for
        every temperature. Albeit the efficiency of the algorithmic procedure
        was not dissected for every temperature, I feel that the speed up
        near criticality is worth the moderate slow down at other temperatures.
