\label{sec:montecarlo}
Monte Carlo simulations are (probalby) named after the casino in Monaco
\cite{NewmanBarkema1999} because both rely on randomness.\\
The idea behind Monte Carlo simulations is to take random samples of
the observable which should be measured and estimate the observable from
this samples. In example the Monte Carlo integration chooses random
points within the bounding box of the fuction and estimates the integral
as the fraction of points below the function and points above muliplied
by the area of the bounding box.\\
%~ To improve this method: \emph{Importance Sampling}
%~ means that one takes more samples where the contribution of the
%~ function to the integral is bigger by using random numbers
%~ distributed........
It is difficult to create random state of the Ising system in
equilibrium, hence one uses \emph{Markov Chains} to generate new states
\(\nu\) from old ones \(\mu\).
It is important that the transition probabilities \(A(\mu \to \nu)\)
obey \emph{Detailed Balance}, that the probability to leave a state is
the same as the probability to enter the state in equilibrium
\(p_\mu A(\mu \to \nu) = p_\nu A(\nu \to \mu)\)
with \(p_\mu\) the probability to be in state \(\mu\), and
\emph{Ergodicity}, that every possible state is reachable from every
other state in finite time. \cite{NewmanBarkema1999} \cite{Katzgraber2011}\\
Otherwise the samples might not be representative for the whole system.

\subsection{Monte Carlo Algorithms}
    \paragraph{Metropolis}
        Let \(\Delta \hat H(\mu,\nu)\) be the energy difference between
        the states \(\mu\) and \(\nu\)
        \begin{equation}
            \Delta \hat H = \hat H(\nu) - \hat H(\mu)
        \end{equation}
        The Metropolis algorithm \cite{Metropolis1953} chooses the transition
        probability like eq. \ref{eq:metropolis}
        \begin{equation}
            A(\mu \to \nu)
            \begin{cases}
                1                                   & \Delta \hat H \le 0 \\
                \exp{\brac{-\beta \Delta \hat H}}   & \Delta \hat H > 0
            \end{cases}
            \label{eq:metropolis}
        \end{equation}
        A Metropolis Monte Carlo simulation of an Ising model will choose a
        random spin, calculate the energy change \(\Delta \hat H\) that would
        result from a flip of and excute the flip with the probability \(A\).
        \cite{NewmanBarkema1999} \cite{Katzgraber2011}

    \paragraph{Wolff}
        Close to the critical temperature \(T_c\) the Metropolis
        gets slower. This is called \emph{critical slowing} down and the
        cause is beyond the scope of this thesis.\\
        Using a cluster algorithm like the Wolff
        algorithm \cite{Wolff1989} speeds things up.
        For an Ising model the Wolff algorithm builds a cluster of sites
        starting with an random seed and adding neighboring sites with
        the same spin with probability
        \(P_{\mathrm{add}} = 1-\exp\brac{-2\beta J}\)
        where \(J\) is the coupling constant (explained in section
        \ref{ssec:isingmodel}). The neighboring sites of the added sites
        are also considered and so forth. When there are no more sites
        to add, the spin of every site in the cluster is flipped
        \cite[S. ??]{NewmanBarkema1999} \cite[S. 151f]{Katzgraber2011}.
        This leads fast to new uncorrelated states at the critical
        temperature because big clusters are flipped. But there are not
        much advantages at high or low temperatures. At low temperatures
        the cluster will consist of almost all sites such that all but
        very few spins will be flipped. At high temperatures the cluster
        will only contain very few sites.
        Both situaitions have no advantage against the Metropolis algorithm.

    \paragraph{Parallel Tempering}
        The main aim is to obtain the critical temperatures
        \(T_c\) for different disturbance paramters \(\sigma\).
        Therefore it is necessary to simulate for many temperatures,
        so that \emph{Parallel Tempering}\footnote{Before R. H.
            Swendsen published this paper, a algorithm \(MC^3\) was
            already published with the same idea. [citation needed]}
        \cite{ParallelTempering1986} is a suited algorithm. Parallel
        Tempering simulates many identical systems at different
        temperatures and periodically swaps the spin configurations
        between two neighboring temperatures with probability \(P\) from
        eq. \ref{eq:partemp} \cite[S. ??]{NewmanBarkema1999} \cite[S. 155ff]{Katzgraber2011}.
        \begin{equation}
            P((E_i,T_i) \to (E_{i+1},T_{i+1})) = \min\brac{1,\exp\brac{\brac{E_{i+1}-E_i}\brac{\frac{1}{T_{i+1}}-\frac{1}{T_i}}}}
            \label{eq:partemp}
        \end{equation}
        This has the advantages that correlation times of single
        temperatures are far smaller because their spin configuration
        get often replaced by other uncorrelated configurations. And in
        many cases the more important advantage is that a system which
        is trapped in a local minimum at a given temperature can travel
        to higher temperatures leave it's local minimum and cool down
        again in a lower minimum.

    \paragraph{Implementation Details}
        Here a mixture of the above three algorithms is used.
        Each sweep \(N\) Metropolis spin flips, one Wolff cluster flip
        and one Parallel Tempering swap are performed.
%~ Monte Carlo Simulationen:\\
    %~ Wolff-Cluster Algorithmus \cite{Wolff1989} (siehe auch \cite[S. xx]{NewmanBarkema1999}),
    %~ Metropolis Sweep \cite{Metropolis1953} (siehe auch \cite[S. xx]{NewmanBarkema1999}),
    %~ Parallel Tempering \cite{ParallelTempering1986} (siehe auch \cite[S. xx]{NewmanBarkema1999} \cite[S. xx]{Katzgraber2011})\\

\subsection{Equilibration- and Autocorrelation Time}
    To generate an equilibrium state one starts with an arbitrary state
    and waits until it is equilibrated. The count of sweeps till
    equilibrium is called \emph{equilibration time} \(t_{eq}\).
    All measurement should start after this time.
    % in meinem fall ... vielleicht bild?
    \\
    Because every state is generated from the state before, measurements
    of subsequent states are correlated. To determine when two states
    are independet, one calculates the autocorrelation function \(\chi(t)\)
    and integrates it to get the autocorrelation time \(\tau = \int \chi(t) \de t\).
    \(\tau\) is an estimate after which time two samples are not
    correlated anymore. \cite[S. ??]{NewmanBarkema1999} \cite[S. 150f]{Katzgraber2011}.
    To make sure that the error is not underestimated one should wait
    \(2\tau\) sweeps between two measurements.

