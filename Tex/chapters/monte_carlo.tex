\subsubsection{Algorithm}
    Monte Carlo simulations are used to obtain any properties of the
    model.
    %Importance Sampling
    %Detailed Balance
    %Ergodicity
    \paragraph{Metropolis}
        A basic Monte Carlo algorithm is the Metropolis algorithm
        \cite{Metropolis1953} ... [einzelheiten]
    \paragraph{Wolff}
        Close to the critical temperature \(T_c\) the Metropolis
        gets slower. Using a cluster algorithm like the Wolff
        algorithm \cite{Wolff1989} speeds things up.
        ... [einzelheiten]
    \paragraph{Parallel Tempering}
        The main aim is to obtain the critical temperatures
        \(T_c\) for different disturbance paramters \(\sigma\).
        Therefore it is necessary to simulate for many temperatures,
        so that \emph{Parallel Tempering}\footnote{Before R. H.
            Swendsen published this paper, a algorithm \(MC^3\) was
            already published with the same idea. [citation needed]}
        \cite{ParallelTempering1986} is a suited algorithm. Parallel
        Tempering simulates many identical systems at different
        temperatures and periodically swaps the spin configurations
        between two neighboring temperatures with probability
        \(\). Some good descriptions are in
        \cite[S. ??]{NewmanBarkema1999} \cite[S. 155ff]{Katzgraber2011}.
    \paragraph{Implementation Details}
        Here a mixture of the above three algorithms is used.
        Each sweep a Metropolis sweep, a Wolff cluster step and a
        Parallel Tempering swap are performed.
    %~ Monte Carlo Simulationen:\\
        %~ Wolff-Cluster Algorithmus \cite{Wolff1989} (siehe auch \cite[S. xx]{NewmanBarkema1999}),
        %~ Metropolis Sweep \cite{Metropolis1953} (siehe auch \cite[S. xx]{NewmanBarkema1999}),
        %~ Parallel Tempering \cite{ParallelTempering1986} (siehe auch \cite[S. xx]{NewmanBarkema1999} \cite[S. xx]{Katzgraber2011})\\
