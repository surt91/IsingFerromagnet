In this thesis the properties of an Ising ferromagnet, which nodes are
moved by an gauss distributed displacement, is studied. The neighbor
relationship is provided by two proximity graphs -- the RNG and the GG.
The coupling constants are distance dependend. This model is within the
universality of the standard Ising ferromagnet. Its critical temperature
decreases with increasing disorder on a RNG. On a GG it first jumps,
increases up to a maximum and decreases after that. The course of the
critical temperatures can be made plausible observing the degree of the
underlying graph, but no explicit formular was found.

\subsection{Outlook}
    The behavior of the normalized critical temperatures in dependence
    on \(\sigma\) of the Relative Neighborhood graph and Gabriel graph
    are qualitative similar to each other. One could study the behavior
    of other proximity graphs like the minimum spanning tree or the
    Delaunay triangulation, to examine if it is also similar. Further
    one could examine other graph ensembles to determine, if this is a
    more universal property.\\
    Also it would be interesting to compare this results with the
    results obtained by the same experiments with \(\alpha = 0\), which
    implies that the coupling constant is independent from the distance
    on the sites. That way the critical temperature would be more
    straight forward normalized by the degree of the graph. This way it
    could be determined if the qualitative similarities of the
    normalized temperatures over the disorder parameter of different
    graph types are related to the choice of the function determining the
    coupling constant.
