In this thesis the properties of an Ising ferromagnet, which nodes are
moved by a Gaussian distributed displacement up to the limiting case of
a Poisson process, is studied with Monte Carlo
Simulations. The neighbor relationship is provided by two proximity graphs
-- the RNG and the GG. These graphs conserve the two-dimensional, planar
character of the lattice. The coupling constants
are distance dependent, with increasing distance the coupling strength
decreases exponentially.
This model is within the universality of the square lattice Ising
ferromagnet. Its critical temperature decreases with increasing disorder
on a RNG. The reason is that the decreasing average coordination number
of the underlying graph lowers the resistance to spin flips within
strongly polarized configurations at low temperatures.
On a GG \(T_c\) first jumps, increases up to a maximum and decreases
afterwards. At high disorder it approaches the limit of randomly generated
graphs of the respective type. The course of the critical temperatures
can be made plausible observing the degree of the underlying graph, an
approximate analytic statement as reason for the observed behavior was
also presented.

\subsection{Outlook}
    The behavior of the normalized critical temperatures in dependence
    on \(\sigma\) of the Relative Neighborhood graph and Gabriel graph
    are qualitatively similar to each other. One could study the behavior
    of other proximity graphs like the minimum spanning tree or the
    Delaunay triangulation to examine if it is also similar.\\
