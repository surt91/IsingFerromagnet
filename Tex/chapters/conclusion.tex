In this thesis the properties of an Ising ferromagnet, which nodes are
moved by a Gaussian distributed displacement up to the limiting case of
a Poisson process, is studied with Monte Carlo
Simulations. The neighbor relationship is provided by two proximity graphs
-- the RNG and the GG. These graphs conserve the two-dimensional, planar
character of the lattice. The coupling constants
are distance dependent, with increasing distance the coupling strength
decreases exponentially.
This model is within the universality of the square lattice Ising
ferromagnet. Its critical temperature decreases with increasing disorder
on a RNG.
On a GG \(T_c\) first jumps, increases up to a maximum and decreases
afterwards. At high disorder it approaches the limit of randomly generated
graphs of the respective type. The course of the critical temperatures
can be made plausible observing the degree of the underlying graph.
The decreasing average coordination number
of the underlying graph lowers the resistance to spin flips within
strongly polarized configurations at low temperatures.
An approximation of the critical temperature from the degree of the graph
an the type of the graph was given.

\subsection{Outlook}
    The curves of an \(T(K)\) plot for the RNG and GG are distinct. One
    could study the behavior of other proximity graphs like the minimum
    spanning tree or the Delaunay triangulation to examine if their critical
    temperatures would fall on other curves or lie on the curve of e.g.\ the RNG.\\
    Also the approach of investigating the Binder Cumulant as a function
    of the correlation length \(g(\xi)\), which is more
    sensitive detecting nonuniversalities\cite{Hartmann2013}, could be considered.\\
    It would also be interesting if one could find a better approximation
    of \(T_c\) considering not only the degree but also other graph
    properties like the mean edge length or edge weight.
