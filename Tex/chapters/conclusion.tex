In this thesis the properties of an Ising ferromagnet, which nodes are
moved by a Gaussian distributed displacement, is studied with Monte Carlo
Simulations. The neighbor relationship is provided by two proximity graphs
-- the RNG and the GG. These graphs conserve the two dimensional character
of the lattice, because their edges do not cross. The coupling constants
are distance dependend, they are getting exponential weaker at higher distances.\\
This model is within the universality of the square lattice Ising
ferromagnet. Its critical temperature decreases with increasing disorder
on a RNG. On a GG it first jumps, increases up to a maximum and decreases
after that. At high disorder it approaches the limit of randomly generated
graphs of the respective type. The course of the critical temperatures
can be made plausible observing the degree of the underlying graph, but
no explicit formular was found.

\subsection{Outlook}
    The behavior of the normalized critical temperatures in dependence
    on \(\sigma\) of the Relative Neighborhood graph and Gabriel graph
    are qualitative similar to each other. One could study the behavior
    of other proximity graphs like the minimum spanning tree or the
    Delaunay triangulation to examine if it is also similar.\\
