The computing capabilities of modern computers enable researchers to
collect and analyse vast amounts of data.
Further statistical systems, for which exist no analytical soultions
or only for very simplified or special cases, can be simulated.
For example to examine phasetransitions, which are defined by the abrupt
change of an observable i.e. the change of the density of water near
boiling or the change of the magnetisation of a ferromagnet near the
Curie temperature. One of the simplest models with a second order
phasetransition is the Ising model \cite{Ising1925} which is a simple
model of a ferromagnet and will be explained in more detail in section
\ref{ssec:isingmodel}. It is analytically solved for two dimensions on
some regular lattices \cite{Onsager1944} \cite{Wannier1945}.
In this thesis it's behavior near the Curie temperature -- also called
critical temperature -- on some irregular lattices corresponding to
proximity graphs (see section \ref{ssec:graphtypes}) will be examined
using the Monte Carlo simulations described in section \ref{sec:montecarlo}.

Unfortunately the memory of any computer is small in comparison with the
thermodynamic limit. This leads to \emph{finite size effects}.
In section \ref{ssec:finitesize} will be discussed how to manage them.\\

Note that in the scope of this thesis the Boltzmann constant \(k_{B}=1\)
for the sake of simplicity.
