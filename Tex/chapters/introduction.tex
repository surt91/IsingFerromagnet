The computing capabilities of modern computers enable researchers to
collect and analyse vast amounts of data.
Further statistical systems, for which exist no analytical soultions
or only for very simplified or special cases, can be simulated.
%~ Simulations also have the advantage to be able to measure some quantities
%~ which are not obtainable by experiment.
For example to examine phasetransitions, which are defined by the abrupt
change of an observable i.e. the change of the magnetisation of a
ferromagnet near the Curie temperature or the change of the density of
water near boiling. Unfortunatly the memory of any computer is small in
comparison with the thermodynamic limit.
%~ Of course it is only possible to simulate systems of finite size because
%~ of the finite memory of computers so the thermodynamic limit can not be
%~ simulated directly.
This leads to \emph{finite size effects}. In section \ref{ssec:finitesize}
will be discussed how to manage them.\\

One of the simplest models with a second order phasetransition is the
Ising model \cite{Ising1925} which is a simple model of a ferromagnet
and will be explained in more detail in section \ref{ssec:isingmodel}.
It is analyticaly solved for two dimensions on some regular lattices
\cite{Onsager1944} \cite{Wannier1945}.
In this thesis the critical behavior on some irregular lattices
corresponding to proximity graphs (see section \ref{ssec:graphtypes})
will be examined using the Monte Carlo simulations described in section
\ref{sec:montecarlo}. Note that in the scope of this thesis
the Boltzmann constant \(k_{B}=1\) for the sake of simplicity.
All source code is available at \url{github [noch hochzuladen]}.
