The computing capabilities of modern computers enable researchers to
collect and analyze vast amounts of data. The prime example for this fact
is the LHC, where petabytes of particle collision data are collected,
processed and discarded or kept every second \cite{LHC}.
But computers are not only able to process measured data, but
also to generate data by simulations of e.g.\ statistical systems, where
the rules for each subsystem are well defined but the behavior of the
whole system is not easy to predict because of the interactions of the
subsystems. For those systems there often exists no analytic solutions
or only for very simplified or special cases. But one can simulate all
the interactions of the subsystems and observe the behavior of the whole
system.
For example to examine phase transitions, which are defined by the abrupt
change of an observable, e.g.\ the change of the density of water near
boiling or the change of the magnetization of a ferromagnet near the
Curie temperature.
Different phases of a material and the transitions between them were
always of greatest interest. The ancient Greek thought that everything
consists of fire, water, air and earth, which are the archetypes of
phases. Still phase transitions are alone on the basis of their ubiquity
important phenomenons. The understanding allows applications from the
refrigerator to shatter resistant mobile phone glass \cite{PJournalGlass}.
One of the simplest models with a second order
phase transition, i.e.\ a continuous transition without discontinuities of
the order parameter, is the Ising model \cite{Ising1925} in \(d \ge 2\)
dimensions. This is a simple
model of a ferromagnet and will be explained in more detail in section
\ref{ssec:isingmodel}. It is analytically solved in two dimensions on
some regular lattices \cite{Onsager1944} \cite{Wannier1945}.
In this thesis its behavior near the critical temperature -- also called
Curie temperature -- and the behavior of the critical temperature itself
on some irregular lattices corresponding to proximity graphs
(see Sec.\ \ref{ssec:graphtypes}) will be examined using the Monte
Carlo simulations described in Sec.\ \ref{ssec:montecarlo}.

Unfortunately the memory of any computer is small in comparison to what
would be needed for a simulation of a system in the
thermodynamic limit. So only a very small number of subsystems can be
simulated in comparison to the actual number of subsystems
of the system in nature. This leads to deviations from the real behavior
of the system in the thermodynamic limit. These deviations are called
\emph{finite size effects} and in Sec.\ \ref{ssec:finitesize} will be
discussed how to manage them.\\

Note that in the scope of this thesis the Boltzmann constant is \(k_{B}=1\)
for the sake of simplicity.
