The computing capabilities of modern computers enable researchers to
collect and analyze vast amounts of data. The prime example for is fact
is the LHC, where petabytes of particle collision data is collected and
processed. But computers are not only able to process measured data, but
also to generate data by simulations of e.g. statistical systems, where
the rules for each subsystem are well defined but the behavior of the
whole system is not easy to predict, beacause of the interactions of the
subsystems. For those systems there often exist no analytic solutions
or only for very simplified or special cases. But one can simulate all
the interactions of the subsystems and observe the behavior of the whole
system.
For example to examine phase transitions, which are defined by the abrupt
change of an observable e.g. the change of the density of water near
boiling or the change of the magnetization of a ferromagnet near the
Curie temperature. One of the simplest models with a second order
phase transition is the Ising model \cite{Ising1925} which is a simple
model of a ferromagnet and will be explained in more detail in section
\ref{ssec:isingmodel}. It is analytically solved for two dimensions on
some regular lattices \cite{Onsager1944} \cite{Wannier1945}.
In this thesis its behavior near the Curie temperature -- also called
critical temperature -- and the behavior of the Curie temperature itself
on some irregular lattices corresponding to proximity graphs
(see section \ref{ssec:graphtypes}) will be examined using the Monte
Carlo simulations described in section \ref{sec:montecarlo}.

Unfortunately the memory of any computer is small in comparison with the
thermodynamic limit. So there can only be a very small number of
subsystems simulated in comparison to the actual number of subsystems
of the system in nature. This leads to deviations from the real behavior
of the system in the thermodynamic limit. These deviations are called
\emph{finite size effects} and in section \ref{ssec:finitesize} will be
discussed how to manage them.\\

Note that in the scope of this thesis the Boltzmann constant \(k_{B}=1\)
for the sake of simplicity.
