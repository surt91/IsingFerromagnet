The computing capabilities of modern computers enable researchers to
collect and analyze vast amounts of data. The prime example for this fact
is the LHC, where petabytes of particle collision data are collected,
processed and discarded or kept every second \cite{LHC}.
But computers are not only able to process measured data, but
also to generate data by simulations of, e.g., statistical systems, where
the rules for each subsystem are well defined but the behavior of the
whole system is not easy to predict because of the nontrivial interactions of the
subsystems. For those systems there often exists no analytic solution
or only for very simplified or special cases. But one can simulate all
the interactions of the subsystems and observe the behavior of the whole
system using computational experiments.
For example one can examine phase transitions, which are defined by the abrupt
change of an observable, e.g., the change of the density of water near
boiling, which is liquid at \(T < T_c\) and gaseous at \(T > T_c\). Also
the change of the magnetization of a ferromagnet near the Curie temperature
is a phase transition from ferromagnetic at \(T < T_c\) to paramagnetic at \(T > T_c\).
This can be observed by heating some refrigerator magnet by a candle --
after this treatment it will no longer stick on the refrigerator.
Different phases of a material and the transitions between them were
always of greatest interest. The ancient Greek thought that everything
consists of fire, water, air and earth, which are the archetypes of
phases. Still phase transitions are important phenomenons, because they
are ubiquitious. The understanding allows applications from the
refrigerator to shatter resistant mobile phone glass \cite{PJournalGlass}.
There are different kinds of phase transitions which are distinguished by
the behavior of their order parameter and characterized by a set of
critical exponents \cite{yeomans}. If it shows a discontinuity at
the phase transition, it is a \emph{first order} phase transition, e.g.\ Water at boiling.
A \emph{second order} phase transition is characterized by a continuous
transition without discontinuities of the order parameter but a discontinuity
in its first derivative.\\
One of the simplest models with a second order phase transition is the
Ising model \cite{Ising1925} in \(d \ge 2\) dimensions. This is a simple
model of a ferromagnet and will be explained in more detail in Sec.\
\ref{ssec:isingmodel}. It is analytically solved in two dimensions on
some regular lattices \cite{Onsager1944} \cite{Wannier1945}.
In this thesis its behavior near the critical temperature -- also called
Curie temperature -- and the behavior of the critical temperature itself
on some irregular lattices corresponding to proximity graphs
(see Sec.\ \ref{ssec:graphtypes}) will be examined using the Monte
Carlo simulations described in Sec.\ \ref{ssec:montecarlo}.\\
Proximity graphs are canditates for ad-hoc networks. As an example of an
ad-hoc network take a wifi network without central devices, but every
participant (e.g.\ a laptop) works as a relay to get a data package to its
destination. To minimize the needed energy the packages will be routed
over small distances from one node to another. This behavior is well mapped
by proximity graphs. They establish a lattice by connecting sites which
are "near" to each other. The exact definition of "near" is dependend on
the proximity graph.\\
Because the Ising model on a regular square lattice is well understood,
the here investigated irregular lattices will be constructed starting from
a square lattice and displacing the sites governed by a parameter \(\sigma\).
Then the influence of \(\sigma\) on the critical temperature will be
analyzed. A crossover of the critical temperature from the square lattice
to the corresponding proximity graph is expected.\\

Unfortunately the memory of any computer is small in comparison to what
would be needed for a simulation of a system in the
thermodynamic limit. So only a very small number of elementary subsystems can be
simulated in comparison to the actual number of elementary subsystems
of the system in nature. This leads to deviations from the real behavior
of the system in the thermodynamic limit. These deviations are called
\emph{finite-size effects} and in Sec.\ \ref{ssec:finitesize} will be
discussed how to manage them.\\

Note that in the scope of this thesis the Boltzmann constant is \(k_{B}=1\)
for the sake of simplicity.
