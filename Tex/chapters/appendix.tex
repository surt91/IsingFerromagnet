\section{Appendix}
\subsection{Finite Size Effects at the Example of the Specific Heat}
\label{appendix:finiteSizeEffects}
    In fig. \ref{fig:smeared_out_appendix} the specific heat
    \begin{equation}
        c = \frac{N}{T^2}\avg{\avg{E^2} - \avg{E}^2}
    \end{equation}
    is plotted for different system sizes. The finite size effects are obvious.
    The divergence is finite, and gets shallower with smaller \(L\). Besides
    the maximum moves away from the critical temperature with smaller \(L\).
    \begin{figure}[htbp]
        \centering
        \includegraphics[width=0.45\textwidth]{plots/Specific_Heat_0}
        \caption[Finite Size Effects by Example of the Specific Heat]
        {
            Effects of different system sizes at \(\sigma = 0\). Dotted lines
            are guides to the eye.
        }
        \label{fig:smeared_out_appendix}
    \end{figure}

\subsection{Course of the Critical Temperature $T_c$ with Fixed Coupling Constants $J$}
\label{appendix:fixedCoupling}
    A quick analysis of this model with fixed coupling constants \(J = 1\)
    (i.e.\ \(\alpha=0\)) is performed. The results are displayed in fig. \ref{fig:Tc_deg_A0}.
    The jump from \(\sigma=0\) to \(\sigma>0\) does not disappear as in fig. \ref{fig:Tc_deg}
    for variable \(J\) (i.e.\ \(\alpha=0.5\)). This suggests, that the
    disappearance of the jump is a random special case for the function
    \(J_{ij}=e^{\alpha(1-d)}\).\\
    The simulations were carried out on \(L \in \{16,32,64\}\) lattices
    for a subset of the \(\sigma\) and \(T\) used in the previous simulation.
    Every other parameter was kept. Also note that the degree \(K\) is
    the same used in \ref{fig:Tc}\subref{sfig:deg:RNG}\subref{sfig:deg:GG}
    because it is obviously independent of \(J\).
    \begin{figure}[htbp]
        \centering
        \subfigure[][]
        {
            \label{sfig:Tc:RNG_A0}
            \includegraphics[width=0.45\textwidth]{plots/RNG_Tc_A0}
        }
        \subfigure[][]
        {
            \label{sfig:Tc:GG_A0}
            \includegraphics[width=0.45\textwidth]{plots/GG_Tc_A0}
        }

        \subfigure[][]
        {
            \label{sfig:Tc_norm_deg:RNG_A0}
            \includegraphics[width=0.45\textwidth]{plots/RNG_Tc_norm_deg_A0}
        }
        \subfigure[][]
        {
            \label{sfig:Tc_norm_deg:GG_A0}
            \includegraphics[width=0.45\textwidth]{plots/GG_Tc_norm_deg_A0}
        }

        \caption[Critical Temperature and Critical Temperature Normalized by Degree of the Graph for Fixed Coupling Constants $J=1$]
        {
            Top: Critical Temperature \(T_c\) of the graph over different
            disorder parameters \(\sigma\) with fixed coupling constants \(J=1\) for
            \subref{sfig:deg:RNG} the RNG and
            \subref{sfig:deg:GG} the GG.\\
            Bottom: Critical temperatures normalized by degree \(K\) over
            different disorder parameters \(\sigma\) with fixed coupling constants \(J=1\) for
            \subref{sfig:Tc_norm_deg:RNG} the RNG and
            \subref{sfig:Tc_norm_deg:GG} the GG.
        }
        \label{fig:Tc_deg_A0}
    \end{figure}\\
