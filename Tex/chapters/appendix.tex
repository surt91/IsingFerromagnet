\section{Finite Size Effects at the Example of the Specific Heat}
\label{appendix:finiteSizeEffects}
    In fig. \ref{fig:smeared_out_appendix} the specific heat
    \[C = \frac{N}{T^2}\avg{\avg{E^2} - \avg{E}^2}\]
    is plotted for different system sizes. The finite size effects are obvious.
    The divergence is finite, and gets shallower with smaller \(L\). Besides
    the maximum moves away from the critical temperature with smaller \(L\).
    \begin{figure}[htbp]
        \centering
        \includegraphics[width=0.45\textwidth]{plots/Specific_Heat_0}
        \caption[Finite Size Effects by Example of the Specific Heat]
        {
            Effects of different system sizes at \(\sigma = 0\). Dotted lines
            are guides to the eye.
        }
        \label{fig:smeared_out_appendix}
    \end{figure}

\section{Finite Size Scaling at the Example of the Susceptibility and Magnetisation}
\label{appendix:finiteSizeScaling}
    In fig. \ref{fig:gettingCrit:appendix}\subref{sfig:gettingCrit:s_0_sus}
    the magnetic susceptibility
    \[\chi = \frac{1}{TN}\avg{\avg{E^2} - \avg{E}^2}\]
    is plotted and in \ref{fig:gettingCrit:appendix}\subref{sfig:gettingCrit:collapse_s_0_sus}
    collapsed to determine the critical exponent \(\beta\).
    Analog for the mean absolute magnetisation \(\avg{\abs{m}}\) in fig \ref{fig:gettingCrit:appendix}
    \subref{sfig:gettingCrit:s_0_meanM}\subref{sfig:gettingCrit:collapse_s_0_meanM}
    to determine the critical exponent \(\gamma\).
    Both are at \(\sigma=0\) and
    both collapses reproduce the  critical exponent \(\nu\) and the
    critical temperature \(T_c\) in good agreement with the before determined
    values from fig. \ref{fig:gettingCrit}\subref{sfig:gettingCrit:collapse_s_0}
    and the analytically known values. This is a good cross check
    for consistency.
    \begin{figure}[htbp]
        \centering
        \subfigure[Susceptibility $\chi$][]
        {
            \label{sfig:gettingCrit:s_0_sus}
            \includegraphics[width=0.47\textwidth]{plots/s_0_sus}
        }
        \subfigure[Finite Size Scaling of the Susceptibility $\chi$][]
        {
            \label{sfig:gettingCrit:collapse_s_0_sus}
            \includegraphics[width=0.47\textwidth]{plots/collapse_s_0_sus}
        }
        \subfigure[Magnetisation $\avg{\abs{m}}$][]
        {
            \label{sfig:gettingCrit:s_0_meanM}
            \includegraphics[width=0.47\textwidth]{plots/s_0_meanM}
        }
        \subfigure[Finite Size Scaling of the Magnetisation $\avg{\abs{m}}$][]
        {
            \label{sfig:gettingCrit:collapse_s_0_meanM}
            \includegraphics[width=0.47\textwidth]{plots/collapse_s_0_meanM}
        }
        \caption[Examples of determining critical temperature and exponents]
        {
            Examples for finite size scaling.
        }
        \label{fig:gettingCrit:appendix}
    \end{figure}
