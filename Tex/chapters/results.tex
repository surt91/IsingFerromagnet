\subsection{Critical Temperature}
    The evaluation of the Binder cumulant's intersections, yields the
    critical temperatures \(T_c\), which are plotted in
    fig. \ref{fig:Tc}\subref{sfig:Tc:RNG}\subref{sfig:Tc:GG}.
    Further in fig. \ref{fig:Tc}\subref{sfig:sumJ:RNG}\subref{sfig:sumJ:GG}
    the mean sum of the coupling constants to all neighbors \(\avg{\sum_{\avg{i,j}} J_{ij}}\)
    is plotted which is evidently correlated with \(T_c\).
    That is plausible, because \(T_c\) is dependend on the coupling
    constant \(J\) (compare \eqref{eq:exactTc}).\\
    It seems reasonable to normalize the values of \(T_{c}\) by
    \(\avg{\sum_{\avg{i,j}} J_{ij}}\) which yields the diagramms fig.
    \ref{fig:Tc}\subref{sfig:Tc_norm:RNG}\subref{sfig:Tc_norm:GG}.
    \begin{figure}[htbp]
        \centering
        \subfigure[][]
        {
            \label{sfig:Tc:RNG}
            \includegraphics[width=0.45\textwidth]{plots/RNG_Tc}
        }
        \subfigure[][]
        {
            \label{sfig:Tc:GG}
            \includegraphics[width=0.45\textwidth]{plots/GG_Tc}
        }

        \subfigure[][]
        {
            \label{sfig:sumJ:RNG}
            \includegraphics[width=0.45\textwidth]{plots/RNG_sumJ}
        }
        \subfigure[][]
        {
            \label{sfig:sumJ:GG}
            \includegraphics[width=0.45\textwidth]{plots/GG_sumJ}
        }

        \subfigure[][]
        {
            \label{sfig:Tc_norm:RNG}
            \includegraphics[width=0.45\textwidth]{plots/RNG_Tc_norm}
        }
        \subfigure[][]
        {
            \label{sfig:Tc_norm:GG}
            \includegraphics[width=0.45\textwidth]{plots/GG_Tc_norm}
        }
        \caption[Critical Temperature over different disturbance parameters]
                {Top: Critical temperatures over different
                 disturbance parameters for
                 \subref{sfig:Tc:RNG} the Relative Neighborhood Graph and
                 \subref{sfig:Tc:GG} the Gabriel Graph.\\
                 Middle: Mean sum of the coupling constants to all
                 neighbors over different disturbance parameters for
                 \subref{sfig:sumJ:RNG} the Relative Neighborhood Graph and
                 \subref{sfig:sumJ:GG} the Gabriel Graph.\\
                 Bottom: Normalized critical temperatures over different
                 disturbance parameters for
                 \subref{sfig:Tc_norm:RNG} the Relative Neighborhood Graph and
                 \subref{sfig:Tc_norm:GG} the Gabriel Graph.
                }
        \label{fig:Tc}
    \end{figure}
    % I don't know what all that means...

\subsection{Critical Exponents}
    For \(\sigma \in \{0,0.1,0.5\}\) a finite size scaling analysis was
    performed to determine the critical exponents \(\beta, \gamma, \nu\)
    using \texttt{autoscale.py} \cite{autoscale2009}. The values for
    \(\sigma = 0\) are analytically known \cite{Pelissetto2002}. The
    values for all other \(\sigma\) should be the same according to ???
    [citiation needed]. Like in tab. \ref{tab:critExp} to see, most values
    are matching the expectations. Most \(\beta\) seem to be a bit too
    big, but they are close enough to the expectations to be explained
    by the fact that small systems (\(L=32,64\)) were used for the
    analysis. [citation needed]
    \begin{table}[htbp]
        \center
        \begin{tabular}{l l l l l}
            \toprule
             & \multicolumn{1}{c}{\(\sigma\)} & \multicolumn{1}{c}{\(\nu\)} & \multicolumn{1}{c}{\(\gamma\)} & \multicolumn{1}{c}{\(\beta\)}\\
            \midrule
            exact (\cite[p. 59]{Pelissetto2002}) & \multicolumn{1}{c}{\(0\)} & \multicolumn{1}{c}{\(1\)} & \multicolumn{1}{c}{\(-\frac{7}{4}\)} & \multicolumn{1}{c}{\(\frac{1}{8}\)}\\
            \midrule
            Gabriel      & 0.0 & 1.008(4) & -1.735(2) & 0.1262(4)\\
                         & 0.1 & 1.02(1)  & -1.744(5) & 0.133(6) \\
                         & 0.5 & 1.009(8) & -1.75(1)  & 0.125(13)\\
            \midrule
            Relative N.  & 0.0 & 1.007(2) & -1.739(2) & 0.130(1) \\
                         & 0.1 & 0.99(1)  & -1.746(5) & 0.133(4) \\
                         & 0.5 & 1.00(2)  & -1.75(2)  & 0.143(13)\\
            \bottomrule
        \end{tabular}
        \caption{Critical exponents for different \(\sigma\)}
        \label{tab:critExp}
    \end{table}

    %~ <J>\(\sigma\), Anmerkung zur Anomalie des GG und der Ähnlichkeit von <J> zu \(T_c\)\\
%~
    %~ Erklärung zu Autokorreltationszeit \(\tau\) und Equilibrierungszeit \(t_eq\)\\
    %~ Darstellung von Suszeptibilität \(\chi\), spezifischer Wärme \(c\), mittlerer Magnetisierung \(<m>\) über Unordnungsparamter \(\sigma\)\\
    %~ Bestimmung der kritischen Punkte\\
        %~ Polyfit 4ten Grades durch Binder Kumulante \cite{Binder1981}\\
        %~ Finite Size Scaling Vergleich der Exponenten (AutoScale \cite{Melchert2009}, Vergleich \cite[S. 59]{Pelissetto2002})\\

