\subsection{Critical Temperature}
    The evaluation of the Binder cumulant's intersections, yields the
    critical temperatures \(T_c\), which are plotted in
    fig. \ref{fig:Tc}\subref{sfig:Tc:RNG}\subref{sfig:Tc:GG}.
    Further in fig. \ref{fig:Tc}\subref{sfig:sumJ:RNG}\subref{sfig:sumJ:GG}
    the mean sum of the coupling constants to all neighbors \(\avg{\sum_{\avg{i,j}} J_{ij}}\)
    is plotted which is evidently correlated with \(T_c\).
    That is plausible, because \(T_c\) is dependend on the coupling
    constant \(J\) (compare \eqref{eq:exactTc}).\\
    It seems reasonable to normalize the values of \(T_{c}\) by
    \(\avg{\sum_{\avg{i,j}} J_{ij}}\) which yields the diagramms fig.
    \ref{fig:Tc}\subref{sfig:Tc_norm:RNG}\subref{sfig:Tc_norm:GG}.
    \begin{figure}[htbp]
        \centering
        \subfigure[][]
        {
            \label{sfig:Tc:RNG}
            \includegraphics[width=0.45\textwidth]{plots/RNG_Tc}
        }
        \subfigure[][]
        {
            \label{sfig:Tc:GG}
            \includegraphics[width=0.45\textwidth]{plots/GG_Tc}
        }

        \subfigure[][]
        {
            \label{sfig:sumJ:RNG}
            \includegraphics[width=0.45\textwidth]{plots/RNG_sumJ}
        }
        \subfigure[][]
        {
            \label{sfig:sumJ:GG}
            \includegraphics[width=0.45\textwidth]{plots/GG_sumJ}
        }

        \subfigure[][]
        {
            \label{sfig:Tc_norm:RNG}
            \includegraphics[width=0.45\textwidth]{plots/RNG_Tc_norm}
        }
        \subfigure[][]
        {
            \label{sfig:Tc_norm:GG}
            \includegraphics[width=0.45\textwidth]{plots/GG_Tc_norm}
        }
        \caption[Critical Temperature over different disorder parameters]
        {
            Top: Critical temperatures over different
            disorder parameters for
            \subref{sfig:Tc:RNG} the Relative Neighborhood Graph and
            \subref{sfig:Tc:GG} the Gabriel Graph.\\
            Middle: Mean sum of the coupling constants to all
            neighbors over different disorder parameters for
            \subref{sfig:sumJ:RNG} the Relative Neighborhood Graph and
            \subref{sfig:sumJ:GG} the Gabriel Graph.\\
            Bottom: Normalized critical temperatures over different
            disorder parameters for
            \subref{sfig:Tc_norm:RNG} the Relative Neighborhood Graph and
            \subref{sfig:Tc_norm:GG} the Gabriel Graph.
        }
        \label{fig:Tc}
    \end{figure}\\
    % I don't know what all that means...
    The Gabriel graph, which is shown on the right side of fig. \ref{fig:Tc},
    jumps from \(\sigma = 0\) to \(\sigma > 0\). This is easily explained
    by the definition of the Gabriel graph and it's aftereffects for
    the transition from \(\sigma = 0\). Visible in fig. \ref{fig:GG_sigma}
    a small change of sigma causes many new egdes to arise\footnote{See also \url{http://www.youtube.com/watch?v=PcVZ2pG11GI} for an animation.}.
    To fully understand this, take four nodes forming a square. The edge
    across does not exist, because the other two nodes are on the edge
    of the lune. Moving one node slightly into the square, causes the lune
    to get smaller, hence no other nodes are inside or on the edge of
    the lune anymore and the edge appears.
    \begin{figure}[htbp]
        \centering
        \subfigure[][]
        {
            \label{sfig:GG_sigma:zero}
            \includegraphics[width=0.40\textwidth]{images/GG/sigma_e0}
        }
        \subfigure[][]
        {
            \label{sfig:GG_sigma:notzero}
            \includegraphics[width=0.40\textwidth]{images/GG/sigma_g0}
        }
        \caption[]
        {
            Gabriel graph with periodic boundary conditions for
                \subref{sfig:GG_sigma:zero} \(\sigma = 0\)
                \subref{sfig:GG_sigma:notzero} \(\sigma = 0.01\)
        }
        \label{fig:GG_sigma}
    \end{figure}\\
    Moreover fig. \ref{fig:Tc}\subref{sfig:Tc_norm:RNG}\subref{sfig:Tc_norm:GG}
    shows monotonically decreasing \(T_c\) with increasing disorder
    parameter \(\sigma\). The forms of both curves are similar, but the
    one for the Relative Neighborhood graph \ref{fig:Tc}\subref{sfig:Tc_norm:RNG}
    is generally lower and spans over a bigger temperature range than
    the curve of the Gabriel graph \ref{fig:Tc}\subref{sfig:Tc_norm:GG}.
    % Warum??? Die eigenschaften des Graphen sollten mit der Normalisierung erledigt sein.
    % RNG: laengere Kanten -> weniger lokale effekte?
    For big \(\sigma\) the curve approaches the limit of randomly
    distributed nodes, hence \(T_c\) is independet of \(\sigma\) for
    \(\sigma >> 1\).\\
    Also both graph types have a plateau at \(0 < \sigma < 0.1\). So
    small disorder has little influence on the normalized critical temperature.
    But once the degree of the graph and hence the mean sum of the coupling
    constants decreases, the normalized critical temperature decreases, too.
    % The author is not sure what it means.

\subsection{Critical Exponents}
    For \(\sigma \in \{0,0.1,0.2,0.3,0.5,1.0\}\) a finite size scaling analysis was
    performed to determine the critical exponents \(\beta, \gamma, \nu\)
    using \texttt{autoscale.py} \cite{autoscale2009}. The values for
    \(\sigma = 0\) are analytically known \cite{Pelissetto2002}. The
    values for all other \(\sigma\) should be the same as for \(\sigma = 0\)
    like mentioned before in section \ref{ssec:finitesize}.
    The values for \(\sigma\) are chosen to represent every interesting
    region from fig. \ref{fig:Tc}\subref{sfig:Tc_norm:RNG}\subref{sfig:Tc_norm:GG}:
    the plateau at small \(\sigma\), the steep decline and the plateau at
    big \(\sigma\).\\
    Like in tab. \ref{tab:critExp} to see, most values
    are matching the expectations. Most \(\beta\) seem to be a bit too
    big, but they are close enough to the expectations to be explained
    by the fact that small systems (\(L=32,64\)) were used for the
    analysis.
    Anyway, two critical exponents are sufficient to determine the universality
    class. Therefore this disordered Ising modell is in the same universality
    class as the standard Ising ferromagnet \cite[p. 145]{Katzgraber2011}.
    \begin{table}[htbp]
        \center
        \begin{tabular}{l l l l l}
            \toprule
             & \multicolumn{1}{c}{\(\sigma\)} & \multicolumn{1}{c}{\(\nu\)} & \multicolumn{1}{c}{\(\gamma\)} & \multicolumn{1}{c}{\(\beta\)}\\
            \midrule
            exact (\cite[p. 59]{Pelissetto2002}) & \multicolumn{1}{c}{\(0\)} & \multicolumn{1}{c}{\(1\)} & \multicolumn{1}{c}{\(-\frac{7}{4}\)} & \multicolumn{1}{c}{\(\frac{1}{8}\)}\\
            \midrule
            Gabriel      & 0.0 & 1.008(4) & -1.735(2) & 0.1262(4)\\
                         & 0.1 & 1.02(1)  & -1.744(5) & 0.133(6) \\
                         & 0.3 & 1.000(5) & -1.724(16)& 0.129(12)\\
                         & 0.5 & 1.009(8) & -1.750(12)& 0.125(13)\\
                         & 1.0 & 1.015(22)& -1.743(17)& 0.123(16)\\
            \midrule
            Relative N.  & 0.0 & 1.007(2) & -1.739(2) & 0.130(1) \\
                         & 0.1 & 0.99(1)  & -1.746(5) & 0.133(4) \\
                         & 0.2 & 1.022(17)& -1.756(14)& 0.123(10)\\
                         & 0.5 & 1.002(17)& -1.750(16)& 0.143(13)\\
                         & 1.0 & 1.011(20)& -1.758(16)& 0.138(13)\\
            \bottomrule
        \end{tabular}
        \caption{Critical exponents for different \(\sigma\)}
        \label{tab:critExp}
    \end{table}

    %~ <J>\(\sigma\), Anmerkung zur Anomalie des GG und der Ähnlichkeit von <J> zu \(T_c\)\\
%~
    %~ Erklärung zu Autokorreltationszeit \(\tau\) und Equilibrierungszeit \(t_eq\)\\
    %~ Darstellung von Suszeptibilität \(\chi\), spezifischer Wärme \(c\), mittlerer Magnetisierung \(<m>\) über Unordnungsparamter \(\sigma\)\\
    %~ Bestimmung der kritischen Punkte\\
        %~ Polyfit 4ten Grades durch Binder Kumulante \cite{Binder1981}\\
        %~ Finite Size Scaling Vergleich der Exponenten (AutoScale \cite{Melchert2009}, Vergleich \cite[S. 59]{Pelissetto2002})\\

