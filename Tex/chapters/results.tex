\subsection{Critical Temperature}
    The evaluation of the intersections of the Binder cumulants, yields
    the

\subsection{Critical Exponents}
    %~ <J>\(\sigma\), Anmerkung zur Anomalie des GG und der Ähnlichkeit von <J> zu \(T_c\)\\
%~
    %~ Erklärung zu Autokorreltationszeit \(\tau\) und Equilibrierungszeit \(t_eq\)\\
    %~ Darstellung von Suszeptibilität \(\chi\), spezifischer Wärme \(c\), mittlerer Magnetisierung \(<m>\) über Unordnungsparamter \(\sigma\)\\
    %~ Bestimmung der kritischen Punkte\\
        %~ Polyfit 4ten Grades durch Binder Kumulante \cite{Binder1981}\\
        %~ Finite Size Scaling Vergleich der Exponenten (AutoScale \cite{Melchert2009}, Vergleich \cite[S. 59]{Pelissetto2002})\\

    \begin{figure}[htbp]
        \centering
        \subfigure{\includegraphics[width=0.45\textwidth]{plots/GG_Tc_norm}}
        \subfigure{\includegraphics[width=0.45\textwidth]{plots/RNG_Tc_norm}}
        \caption[Critical Temperature over different disturbance parameters]
                {normalized critical temperatures over different disturbance parameters}
        \label{fig:Tc}
    \end{figure}
