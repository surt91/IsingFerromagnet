\documentclass{beamer}
\usetheme{Hannover}  %% Themenwahl

\usepackage[utf8]{inputenc}
\usepackage[T1]{fontenc}

\usepackage[english,ngerman]{babel}

\usepackage{graphicx}
\usepackage{upgreek}
\usepackage{float}
\usepackage{units}
\usepackage{url}

\usepackage{amsmath}
\usepackage{amssymb}
\usepackage{amsfonts}

\usepackage{longtable}

\usepackage{ae}
\usepackage{booktabs}

\usepackage{tikz}
\usetikzlibrary{patterns}

%\abs{Ausdruck} %Betragsstriche, die skalieren - abgekürzt
\newcommand{\abs}[1]{\ensuremath{\left\vert#1\right\vert}}
% und das gleiche füur große Klammern
\newcommand{\brac}[1]{\ensuremath{\left(#1\right)}}
% Erwartungswert skalierend
\newcommand{\avg}[1]{\left< #1 \right>}
% ein nicht kursives d für Ableitungen/Integrale, mit etwas Platz davor, um sich etwas abzusetzten
\newcommand{\de}{\ensuremath{\,\mathrm{d}}}
% Für Einheiten: schreibt sie nicht kursiv und lässt etwas Platz zur Zahl vorher
\newcommand{\eh}[1]{\ensuremath{\,\mathrm{#1}}}
% einfaches Gradzeichen
\newcommand{\gr}{\ensuremath{^{\circ}}}
% Fehlerfortpfanzung
% dy/dz * delta z
\newcommand{\fehler}[2]%
{\ensuremath{\abs{\frac{\partial #1}{\partial #2}}\cdot \Delta #2}}

\title{Ising-Ferromagnet auf Ad-Hoc Netzwerken}
\author{Hendrik Schawe}
\date{\today}

\begin{document}

\maketitle
\frame{\tableofcontents[pausesections]}

\section{Model}
    \subsection{Ising Ferromagnet}
        \begin{frame}
            \frametitle{Hamiltonian}
            \begin{equation}
                H = - \sum_{\avg{i,j}}J_{ij}s_{i}s_{j}.
            \end{equation}
            (Ref.\ \cite{Ising1925})
        \end{frame}
    \subsection{Proximity Graphs}
        \begin{frame}
            \frametitle{Relative Neighborhood Graph}
        \end{frame}
        \begin{frame}
            \frametitle{Gabriel Graph}
        \end{frame}

\section{References}
    \begin{frame}[allowframebreaks]
        \bibliography{lit}
        \bibliographystyle{amsplain}
    \end{frame}

\end{document}
