\subsection{Modell}
    Ising Modell \(\hat{H} = \sum_{<i,j>}J_{ij}s_{i}s_{j}\)\\
    zufälliges Gitter: Gaußverteilt\\
    Kopplungskonstanten abstandsabhängig: \(J_{ij}=e^{(1-\alpha)d_{ij}}\)\\
    Monte Carlo Simulationen:\\
        Wolff-Cluster Algorithmus \cite{Wolff1989} (siehe auch \cite[S. xx]{NewmanBarkema1999}),
        Metropolis Sweep \cite{Metropolis1953} (siehe auch \cite[S. xx]{NewmanBarkema1999}),
        Parallel Tempering \cite{ParallelTempering1986} (siehe auch \cite[S. xx]{NewmanBarkema1999} \cite[S. xx]{Katzgraber2011})\\
        Erklärung zu Autokorreltationszeit \(\tau\) und Equilibrierungszeit \(t_eq\)\\
        Darstellung von Suszeptibilität \(\chi\), spezifischer Wärme \(c\), mittlerer Magnetisierung \(<m>\) über Unordnungsparamter \(\sigma\)\\
    Relative Neighborhood Graph \cite{Toussaint1980}\\
    Gabriel Graph \cite{Gabriel1969}\\
        Cell Algorithmus ohne Delaunay Triangulation \cite{RNGCell}\\
        <J>(\sigma), Anmerkung zur Anomalie des GG und der ähnlichkeit zu \(T_c\)\\
    Bestimmung der kritischen Punkte\\
        Polyfit 4ten Grades durch Binder Kumulante \cite{Binder1981}\\
        Finite Size Scaling Vergleich der Exponenten (AutoScale \cite{Melchert2009}, Vergleich \cite[S. 59]{Pelissetto2002})\\
