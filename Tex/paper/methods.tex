The results are obtained through Monte Carlo simulations.
For each sweep \(N\) single spin flip Metropolis updates, one
Wolff cluster update and one parallel tempering swap are
performed. These algorithms will be outlined short in the following.\\
Because it is not known before, where the critical temperatures
\(T_c\) are located, the Wolff cluster algorithm is used for
every temperature. Albeit the efficiency of the algorithmic procedure
was not dissected for every temperature, the speed up
near criticality should be worth the moderate slow down at other temperatures.

\subsection{Single Spin Flip Metropolis Update}
    A \emph{Metropolis} Monte Carlo \cite{Metropolis1953} simulation of an
    Ising model will choose a random spin, calculate the energy change
    \begin{equation}
        \Delta H = H_\nu - H_\mu
        \label{eq:dH}
    \end{equation}
    that would result from a flip of that spin. Where \(\mu\) is the
    state before the flip and \(\nu\) after the flip. The flip is then executed
    with the probability
    \begin{equation}
        A(\mu \to \nu) =
        \begin{cases}
            1                            & \Delta H \le 0 \\
            \exp{\brac{-\beta \Delta H}} & \Delta H > 0
        \end{cases}.
        \label{eq:metropolis}
    \end{equation}
    see Refs.\ \cite{NewmanBarkema1999} \cite{Katzgraber2011}.
    So if a transition lowers the energy it will be always done. This
    results in a high ratio between flipped spins and chosen spins.
    Therefore it minimizes the calculations needed for a change of
    the state. Also note that \(\Delta H\) is easy to calculate,
    because it is only affected by the spin of the neighbors of the
    chosen site.

\subsection{Wolff Cluster Update}
\label{sssec:wolff}
    Close to the critical temperature \(T_c\) the efficiency of the
    single spin flip Metropolis update decreases significantly, i.e.\ the
    autocorrelation time \(\tau\) diverges.
    This is called \emph{critical slowing} down.\\
    In order to circumvent this, a cluster algorithm like the \emph{Wolff}
    algorithm \cite{Wolff1989} can be used.
    For an Ising model the Wolff algorithm builds a cluster of sites
    starting with a random site and adding neighboring sites exhibiting the
    same spin orientation with probability
    \begin{equation}
        P_{\mathrm{add}} = 1-\exp\brac{-2\beta J},
        \label{eq:wolffAdd}
    \end{equation}
    where \(J\) is the coupling constant.
    For every site that is added, the neighboring sites of it are
    also considered for addition. (In the case that they are added,
    they are "added sites" and thus their neighbors get a chance to be
    added too.)
    This procedure continues until there are no more sites to add.
    Then the spin of every site in the cluster is flipped
    \cite[p. 91ff]{NewmanBarkema1999} \cite[p. 151f]{Katzgraber2011}.
    This leads fast to new uncorrelated states at the critical
    temperature because big clusters are flipped. But there are not
    much advantages at high or low temperatures. At high temperatures the cluster
    will only contain very few sites, so only few spins will be flipped.
    At low temperatures the cluster will consist of almost all sites
    such that all but very few spins will be flipped. Through this effect the equilibrium
    will be reached sooner at low temperatures than with the Metropolis
    algorithm, but after equilibration the Wolff cluster update has no
    advantage compared to the Metropolis at high or low temperatures.\\
    So one would activate this algorithm near the critical temperature
    but would use a simple Metropolis algorithm at high and low temperatures.

\subsection{Parallel Tempering}
    In simulations using \emph{parallel tempering} \cite{ParallelTempering1986}
    many identical systems at different temperatures are simulated and
    the spin configurations between two neighboring temperatures are
    swapped periodically with probability \cite[p. 169ff]{NewmanBarkema1999} \cite[p. 155ff]{Katzgraber2011}
    \begin{align}
        \nonumber
        P_{\nu,\nu+1}&(S_\nu \leftrightarrow S_{\nu+1}) \\
        &= \min\brac{1,\exp\brac{\brac{E_{\nu+1}-E_\nu}\brac{\frac{1}{T_{\nu+1}}-\frac{1}{T_\nu}}}},
        \label{eq:partemp}
    \end{align}
    as schematically pictured in Fig.\ \ref{fig:parTemp}\subref{sfig:parTemp:schema}.
    This has the advantage that correlation times of single
    temperatures are far smaller, because their spin configuration
    often gets replaced by another uncorrelated configuration. In
    many cases the more important advantage is that a system, which
    is trapped in a local minimum at a given temperature, can travel
    to higher temperatures, leave its local minimum and cool down
    again in a lower minimum. If a system is trapped in such a
    metastable state, ergodicity is not guaranteed anymore.
    This is schematically pictured in Fig.\ \ref{fig:parTemp}\subref{sfig:parTemp:E}.
    \begin{figure}[htbp]
        \centering
        \subfigure[Schematic Diagramm of the Algorithm][]{
            \label{sfig:parTemp:schema}
            \includegraphics[width=0.22\textwidth]{images/parTempSchema}
        }
        \subfigure[Schematic Diagramm of the Energy Landscape][]{
            \label{sfig:parTemp:E}
            \includegraphics[width=0.22\textwidth]{images/parTempE}
        }
        \caption[Visualisation of the Parallel Tempering Algorithm]
        {
            \subref{sfig:parTemp:schema} Schematic representation of
            the swapping of spin configurations of different simulations \(S_i\)
            between temperatures.\\
            \subref{sfig:parTemp:E} Sketch of an energy landscape, where
            the state of the system (filled circle) is trapped in an local
            minimum. At low temperatures it is very unlikley that it
            overcomes the energy barrier \(\Delta E\) to the minimum.
            After a swap to higher energies, the barrier can be overcome
            and after a swap to lower energies again, the state in
            the minimum can be reached (open circle).
        }
        \label{fig:parTemp}
    \end{figure}\\
    In the case of a ferromagnetic Ising model the risk to get trapped
    in a local energy minimum is very low. In the scope of this thesis it is benefical
    to use parallel tempering, because one has to simulate at many temperatures
    to determine the critical temperature. The additional calculations
    to determine whether to swap configurations or not, are small in
    comparison with the much lower autocorrelation times which are
    achieved with parallel tempering.
