The results are obtained through Monte Carlo simulations.
For each sweep \(N\) single spin flip Metropolis updates \cite{Metropolis1953}, one
Wolff cluster update \cite{Wolff1989} and one parallel tempering swap \cite{ParallelTempering1986} are
performed. The latter two algorithms will be outlined short in the following.\\
Because it is not known before, where the critical temperatures
\(T_c\) are located, the Wolff cluster algorithm is used for
every temperature. Albeit the efficiency of the algorithmic procedure
was not dissected for every temperature, the speed up
near criticality should be worth the moderate slow down at other temperatures.

\begin{figure}[htb]
    \centering
    \includegraphics[width=0.47\textwidth]{plots/collapse_s_1_sus}
    \caption[Examples of Determining Critical Temperature and Exponents]
    {
        Examples for the method of FSS, which was used
        to determine the critical exponents \(\nu, \gamma, \beta\) and
        the critical temperature \(T_c\).
        data collapse, to determine \(\gamma\).
    }
    \label{fig:gettingCrit2}
\end{figure}

\begin{table}[htbp]
    \center
    \begin{tabular}{l l l l l l}
        \toprule
         & \multicolumn{1}{c}{\(\sigma\)} & \multicolumn{1}{c}{\(T_c\)} & \multicolumn{1}{c}{\(\nu\)} & \multicolumn{1}{c}{\(\gamma\)} & \multicolumn{1}{c}{\(\beta\)}\\
        \midrule
        exact (\cite[p. 59]{Pelissetto2002}) & \multicolumn{1}{c}{\(0\)} & 2.2691... & \multicolumn{1}{c}{\(1\)} & \multicolumn{1}{c}{\(\frac{7}{4}\)} & \multicolumn{1}{c}{\(\frac{1}{8}\)}\\
        \midrule
        RNG          & 0.0 & 2.2688(7) & 0.997(11)& 1.737(3) & 0.131(1) \\
                     & 0.1 & 2.2053(6) & 0.994(10)& 1.745(5) & 0.128(3) \\
                     & 0.2 & 1.6265(19)& 1.017(16)& 1.758(14)& 0.157(12)\\
                     & 0.5 & 1.2825(9) & 1.011(18)& 1.746(16)& 0.146(13)\\
                     & 1.0 & 1.2125(5) & 1.010(2) & 1.764(14)& 0.133(12)\\
        \midrule
        GG           & 0.0 & 2.2688(6) & 0.999(13)& 1.739(3) & 0.128(1)\\
                     & 0.1 & 2.8944(43)& 1.001(12)& 1.745(6) & 0.135(6) \\
                     & 0.3 & 2.5281(27)& 1.034(29)& 1.718(15)& 0.118(11)\\
                     & 0.5 & 2.2388(9) & 1.006(5) & 1.747(12)& 0.125(11)\\
                     & 1.0 & 2.1275(20)& 1.036(31)& 1.750(16)& 0.125(15)\\

        \bottomrule
    \end{tabular}
    \caption[Critical Exponents for Different $\sigma$]{
        Critical exponents for different values of \(\sigma\). A finite size
        scaling analysis was performed to determine the critical
        exponents \(\beta, \gamma, \nu\) and the critical temperature
        \(T_c\). The errors for \(\beta\) and \(\gamma\) are estimated
        by \texttt{autoscale.py} \cite{autoscale2009}. The errors of
        \(\nu\) and \(T_c\) are the standard deviation of three obtained
        values through different collapses (see text).
    }
    \label{tab:critExp}
\end{table}

\begin{figure}[hbtp]
    \centering
    \includegraphics[width=.45\textwidth]{plots/Binder_over_Correlation_All}
    \caption[Binder Cumulant over two-point finite-size correlation function divided by system size]
    {
        Binder Cumulant over two-point finite-size correlation function
        divided by system size. The datapoints for different system sizes
        are drawn with different symbols, while the datapoints for different
        graph ensembles are drawn with different colors.
        The inset zooms into the plot to show that
        all datapoints fall within errorbars onto one curve.
    }
    \label{fig:binderOverCorr}
\end{figure}

\begin{figure*}[hbtp]
    \centering
    \includegraphics[scale=0.6]{plots/RNG_Tc} \quad
    \includegraphics[scale=0.6]{plots/GG_Tc} \quad
    \includegraphics[scale=0.6]{plots/Tc_K_goodChi}
    \caption[Critical Temperature over Different Disorder Parameters]
    {
        Critical temperatures \(T_c\) over different
        disorder parameters \(\sigma\) for
        Interesting points are the jump and rise of the GG at small
        \(\sigma\) and the plateau on the RNG for small \(\sigma\).
        Further \(T_c\)  is plotted as a function of the degree.
        The dotted lines are fits to a power law function \(T_c = aK^b\)
        ignoring the "bulk" of points in the lower left and upper right, where
        deviations from the power law become obvious.
    }
    \label{fig:Tc}
\end{figure*}
