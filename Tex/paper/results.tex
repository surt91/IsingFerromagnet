\label{sec:results}
The results are obtained through Monte Carlo simulations. For each
sweep \(N\) single spin flip Metropolis updates \cite{Metropolis1953},
one Wolff cluster update \cite{Wolff1989} and one
parallel tempering swap \cite{ParallelTempering1986} are performed.
For every system size $L \in \{16,32,64,128\}$ and 19 different
$\sigma \in [0,1.2]$ on $100$ different realizations at least $5000$
independent measurements are taken in equilibrium.\\
Since it is not known beforehand, where the critical temperatures \(T_c\)
are located, the Wolff cluster algorithm is used at every
temperature. Albeit the efficiency of the algorithmic procedure was
not examined for every temperature, the speed up near criticality
should be worth the moderate slow down at other temperatures.\\

It is not expected that the universality class depends on the graph
structure unless the dimension is changed.
To proof this, the critical exponents are determined by a finite-size
scaling (FSS) analysis using the well kown scaling bahaviors of the
magnetization, the susceptiblity and the Binder parameter
in an automated and reproducable way using the
method from Ref.~\cite{autoscale2009}.
In Fig.~\ref{fig:collapse}
\begin{figure}[htb]
    \includegraphics[scale=1]{plots/col_s1_sus}
    \caption[Examples of Determining Critical Temperature and Exponents]
    {
        Example for a collapse of the susceptiblity \(\chi\) on a RNG at
        \(\sigma = 1\) to get the corresponding critical exponent $\gamma$.
        The same procedure is used on the Binder parameter $g$ and the
        magnetization $|m|$ to to determine the critical exponents
        \(\nu\) and \(\beta\) and the critical temperature \(T_c\).
        The inset shows the same data with unscaled axes.
    }
    \label{fig:collapse}
\end{figure}
a data collapse at $\sigma = 1$ for the susceptiblity
\[\chi = \frac{N}{T}\brac{\avgR{\avg{m^2}}-\avgR{\avg{m}}^2}\]
with $\avg{\cdot}$ being the thermal average and $\avgR{\cdot}$ being the
average over the random realizations, is plotted.
All obtained values for different $\sigma$ are visualized in Tab.~\ref{tab:critExp}
\begin{table}[htb]
    \begin{ruledtabular}
        \begin{tabular}{l l l l l l}
             & \multicolumn{1}{c}{\(\sigma\)} & \multicolumn{1}{c}{\(T_c\)} & \multicolumn{1}{c}{\(\nu\)} & \multicolumn{1}{c}{\(\gamma\)} & \multicolumn{1}{c}{\(\beta\)}\\
            %~ \colrule
            \hline
            exact        & 0   & 2.2691... & \(1\)    & \(1.75\) & \(0.125\)\\
            %~ \colrule
            \hline
            RNG          & 0.0 & 2.2688(7) & 1.00(1) & 1.741(2) & 0.125(1) \\
                         & 0.1 & 2.2053(6) & 0.99(1) & 1.745(5) & 0.128(3) \\
                         & 0.2 & 1.6265(19)& 1.02(2) & 1.758(14)& 0.122(7)\\
                         & 0.5 & 1.2825(9) & 1.01(2) & 1.746(16)& 0.128(8)\\
                         & 1.0 & 1.2125(5) & 1.01(1) & 1.764(14)& 0.125(6)\\
            %~ \colrule
            \hline
            GG           & 0.0 & 2.2688(6) & 1.00(1) & 1.738(22)& 0.127(4)\\
                         & 0.1 & 2.8944(43)& 1.00(1) & 1.745(6) & 0.135(6) \\
                         & 0.3 & 2.5281(27)& 1.03(3) & 1.718(15)& 0.118(11)\\
                         & 0.5 & 2.2388(9) & 1.01(1) & 1.747(12)& 0.125(11)\\
                         & 1.0 & 2.1275(20)& 1.04(3) & 1.750(16)& 0.125(15)\\
        \end{tabular}
    \end{ruledtabular}
    \caption[Critical Exponents for Different $\sigma$]{
        Critical exponents for different values of \(\sigma\). A finite size
        scaling analysis was performed to determine the critical
        exponents \(\beta, \gamma, \nu\) and the critical temperature
        \(T_c\). The errors for \(\beta\) and \(\gamma\) are estimated
        with the method from Ref.~\cite{autoscale2009}. The errors of
        \(\nu\) and \(T_c\) are the standard deviation of three obtained
        values through different collapses. The critical exponents
        are in reasonable agreement with the exact known values for the 2D Ising
        universality class. The critical temperatures $T_c$ are shifting
        as expected.
    }
    \label{tab:critExp}
\end{table}
and are within errorbars consistent with the exactly known values for the
square lattice IFM.
A newer method \cite{chiCollapse,Hartmann2013} to test whether two models are in
the same universality class is the analysis of the two-point finite-size
correlation function
\[\xi = \frac{1}{2\sin(\abs{\vec{k}_{\mathrm{min}}}/2)} \sqrt{\frac{\chi(\vec{0})}{\chi(\vec{k}_{\mathrm{min}})} - 1}\]
where $\vec{k}_{min}=(2\pi / L, 0)$ and $\chi(\vec{k})$ is the wave vector
dependent susceptiblity
\[\chi(\vec{k}) = \avgR{\avg{\abs{\brac{\frac{1}{N}\sum_j s_j \exp(ikx_j)}^2}}}\]
which is in this case only evaluated over one direction thus $x_j$ is the
position of the node in the x-direction.

\begin{figure}[htbp]
    \centering
    \includegraphics[scale=1]{plots/Binder_over_Correlation_All}
    \caption[Binder Cumulant over two-point finite-size correlation function divided by system size]
    {
        Binder Cumulant over two-point finite-size correlation function
        divided by system size. The datapoints for different system sizes
        are drawn with different symbols, while the datapoints for different
        graph ensembles are drawn with different colors. Anyway, all points
        fall on one curve within errorbars.
        %~ The inset zooms into the plot to show that
        %~ all datapoints fall within errorbars onto one curve.
    }
    \label{fig:binderOverCorr}
\end{figure}
If the Binder parameter
\[g = \frac{3}{2}\brac{1-\frac{\avgR{\avg{m^4}}}{3\avgR{\avg{m^2}^2}}}\]
is plotted over $\xi$ for different $\sigma$, $L$ and $T$ as shown in
Fig.~\ref{fig:binderOverCorr} all data points fall on the
same curve, which confirms that this IFM on proximity graphs is in the
same universality class as on the square lattice, i.e.~$\sigma=0$.

This study examines the critical temperatures $T_c$ for different $\sigma$.
Therefore the knowledge that the Binder parameters $g$ for different
system sizes $L$ intersect at $T_c$ \cite{Binder1981} as shown in Fig.~\ref{sfig:Tc:binder}
%~ \begin{figure*}[hbtp]
    %~ \centering
    %~ \subfigure{
        %~ \label{sfig:Tc:binder}
        %~ \includegraphics[scale=1]{plots/binder}
    %~ }
    %~ \subfigure{
        %~ \label{sfig:Tc:RNG}
        %~ \includegraphics[scale=1]{plots/Tc_RNG}
    %~ }
    %~ \subfigure{
        %~ \label{sfig:Tc:GG}
        %~ \includegraphics[scale=1]{plots/Tc_GG}
    %~ }
    %~ \caption[Critical Temperature over Different Disorder Parameters]
    %~ {
        %~ Critical temperatures \(T_c\) are obtained by finding the intersections
        %~ of the Binder parameters $g$ for different system sizes $L$.
        %~ They are plotted over different disorder parameters \(\sigma\).
        %~ Interesting points are the jump and rise of the GG at small
        %~ \(\sigma\) and the plateau on the RNG for small \(\sigma\).
    %~ }
    %~ \label{fig:Tc}
%~ \end{figure*}
\begin{figure*}[hbtp]
    \subfigure[]{
        \label{sfig:Tc:binder}
        \includegraphics[scale=1]{plots/binder}
    }
    \subfigure[]{
        \label{sfig:Tc:RNG}
        \label{sfig:Tc:GG}
        \includegraphics[scale=1]{plots/Tc_both}
    }
    \subfigure[]{
        \label{fig:TcK}
        \includegraphics[scale=1]{plots/TcK}
    }
    \caption[Critical Temperature over Different Disorder Parameters]
    {
        \subref{sfig:Tc:binder} Critical temperatures \(T_c\) are obtained by finding the intersections
        of the Binder parameters $g$ for different system sizes $L$.
        \subref{sfig:Tc:RNG} They are plotted over different disorder parameters \(\sigma\).
        Interesting points are the jump and rise of the GG at small
        \(\sigma\) and the plateau on the RNG for small \(\sigma\).
        \subref{fig:TcK} \(T_c\) is plotted in a log-log plot as a function of the degree $K$.
        The lines are fits to a power law function \(T_c = aK^b\)
        ignoring the ``bulk" of points in the lower left and upper right, where
        deviations from the power law become obvious.
        For the RNG the fit yields $b = 1.349(2)$ with $\chi_\mathrm{red}^2 = 1.9$
        and for the GG  $b = 1.300(5)$ with $\chi_\mathrm{red}^2 = 0.8$
    }
    \label{fig:Tc}
\end{figure*}
is used.
Since \(g\) is only measured for discrete values of \(T\),
the points have to be interpolated to find the intersection. Therefore
a \emph{cubic spline} \cite{press2007numerical} interpolation %footnote{created using the \texttt{scipy.interpolate} tools \cite{scipy2001}}
is calculated for the measured points.
%~ Cubic spline interpolation is a piece-wise fitting of polynoms of
%~ degree three which are joined under the condition to be at least two
%~ times continuously differentiable.
%~ This interpolation type has the
%~ advantage that it is only influenced by local points so that the
%~ plateaus at low and high \(T\) do not influence the interpolation in
%~ the vicinity of \(T_c\) -- in contrast to, e.g.\ an polynom fit of
%~ degree 4, which has to be restricted to the vincinity of \(T_c\) to
%~ yield meaningful results.
An even better method would be the
\emph{multiple histogram} method \cite[p. 219ff]{NewmanBarkema1999}.
But the simple cubic spline interpolation seems to deliver results that
are good enough, because the temperatures of the measurements are
sufficiently near to each other. Also a cubic spline interpolation is
far easier to obtain.

The first oddity in Fig.~\ref{sfig:Tc:GG} which needs to be addressed is
the jump of $T_c$ close to $\sigma = 0$, where it jumps from $T_c(0) = 2.269(1)$
to $T_c(0.03) = 2.851(1)$ at the next measured temperature. This is easily explainable,
because the square lattice at $\sigma = 0$ is a very special case for the
GG and at smallest deviations from the square lattice new edges arise in
the GG, which lead to a stronger coupling of spins such that the system stays
ferromagnetic at higher temperatures.

For large values of $\sigma$ the configurations approach the limit of a
Poisson point process, hence $T_c$ approaches its limit value.

The critical temperature $T_c$ decreases with increasing $\sigma$, which
is expected, since the number of edges decreases, which can be seen in
Fig.~\ref{fig:RNG_sigma} and with loosely coupled spins, the temperature
needed to disturb the ferromagnetic phase is lower than with strong coupled
spins. To quantify the influence of this, $T_c$ is plotted over the
degree $K$, i.e.~the mean number of neighbors each spin has, in Fig.~\ref{fig:TcK}.
For a given graph type, $T_c$ can be derived approximately by a power law
from $K$. But there are deviations from this behavior, which reflect
at least the influence of the coupling constant $J$.
