We perform Monte Carlo simulations to determine the critical temperatures
of Ising Ferromagnets (IFM) on proximity graphs with different underlying
node sets, which can be governed by a parameter $\sigma$.
The coupling strengths arise from the Euclidean distance
between coupled spins.
The simulations are carried out on graphs with \(N=16^{2}\) to \(N=128^{2}\)
nodes utilizing the Wolff cluster algorithm and parallel tempering in a
wide temperature range around the critical point to measure
the Binder cumulant as means to obtain the critical temperature.
The critical temperatures are shown to depend mainly on the average degree
and the type of the underlying proximity graph.
We further verify using finite-size scaling methods that the IFM on proximity
graphs is in the same universality class as the IFM on a two-dimensional
square lattice.
