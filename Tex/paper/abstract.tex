I perform Monte Carlo simulations using a combination of the single spin
flip Metropolis, Wolff Cluster update and Parallel Tempering algorithms
to study determine the critical temperatures of an Ising Ferromagnet on
the Relative Neighborhood graph and Gabriel graph which are subgraphs of
the Delaunay Triangulation and proximity graphs. The graphs are derived
from square lattices which nodes are displaced by a Gaussian distributed
random variable. The coupling is set to be depended on the distance of
the nodes. The deviation of the proximity graph from a square lattice is
governed by the width of the Gaussian distribution \(\sigma\). Because
only the lattice structure and coupling is changed, it is expected that
this model lies within the square lattice Ising Ferromagnet universality
class. This expectation is confirmed.
The critical temperatures are shown to depend mainly on the average degree
and the type of the underlying proximity graph.
