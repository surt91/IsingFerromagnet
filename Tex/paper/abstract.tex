We perform Monte Carlo simulations to determine the critical temperatures
of an Ising Ferromagnet (IFM) coupled to different types of 2D proximity
graphs.
%~ In particular we consider Relative Neighborhood graphs and Gabriel graphs
%~ which are subgraphs of the Delaunay Triangulation for a given set of points.
The graphs are derived from square lattices where nodes are displaced by
a Gaussian distributed random variable.
%~ The limit of maximal disorder thus corresponds to a Poisson point process.
The deviation of the proximity graph from a square lattice is
governed by the width \(\sigma\) of the Gaussian distribution.
In our model, the coupling strength depends on the euclidian distance
between the coupled spins.
The critical temperatures are shown to depend mainly on the average degree
and the type of the underlying proximity graph.
We further verify that the model lies within the universality class of
the 2D IFM.
