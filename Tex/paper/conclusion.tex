\label{sec:conclusion}
This study shows that universality of the Ising model is preserved for
irregular graphs on a wide range of configurations which are intermediate
between the square lattice ising model and a random Poisson point process.
Also the critical temperature was measured over this range and determined
that it shows in first approximation a power law behavior with the mean
number of neighbors as the base and an exponent dependent on the
graph type that defines the neighbor relationships.

[ideas -- have to be formulated better]

The next step would be the extension of this model to three and higher
dimensions, where the used proximity graphs are still well defined. It
would also be worthwhile to study properties of the graph ensembles itself,
because -- to our knowledge -- they are mostly studied in 2D \cite{RNGCell}.
