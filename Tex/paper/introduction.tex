The Ising model of ferromagnetism \cite{Ising1925} is one of the
most extensively studied models in statistical mechanics, because it
features a continuous phase transition while being simple.
The model consists of discrete variables, called spins, which can be in
one of two different states.
The neighbor relationships, which define the interactions of spins, can be expressed by graphs and
thus there is a constant interest in the critical behavior of this model
on different graph ensembles. The research provides the whole range
from analytical results for regular lattices \cite{Onsager1944,Wannier1945}
to numerical results for complex networks \cite{Aleksiejuk2002260,Herrero2002SmallWorld,Herrero2004ScaleFree,Herrero2015ScaleFree}.

An important graph is the Delaunay triangulation that finds application
in, e.g., finite volume methods.
For a system of randomly placed spins on a two-dimensional surface
with neighbor relationships derived from a Delaunay triangulation,
the critical temperature has been obtained
and it has been confirmed that it is in the same universality class as the
two-dimensional Ising model on a square lattice \cite{Janke1994,Lima2000,Lima2008}.
This article extends the former research to other
irregular lattices, namely the Relative Neighborhood graph (RNG) and
Gabriel graph (GG), which are subgraphs of the Delaunay triangulation
and belong to the family of \emph{proximity graphs}.
%~ Since in proximity graphs nodes which are
%~ spatially close to each other are connected, they are well suited
%~ to generalize problems defined on regular lattices with nearest neighbor
%~ relationships.
The objective of proximity graphs is to connect nodes which are
spatially close to each other, hence they are suited
to generalize problems defined on regular lattices with nearest neighbor
relationships.
This family of graphs finds application in geographic variation studies in
biology \cite{Sokal1978,Sokal1980,Selander1975}, as potential candidates for ``virtual backbones'' in ad-hoc
networks \cite{Kuhn2003,Bose2001,Santi2005,Karp2000,jennings2002topology}
and in machine learning and pattern recognition \cite{bhattacharya1981application}.
Also, their statistical properties have been under scrutiny recently \cite{RNGCell,norrenbrock2014percolation}.

Here, we consider an ensemble depending on a tunable parameter $\sigma$.
By changing this parameter we can interpolate the configurations
between nodes on a square lattice and nodes distributed by a Poisson point process.
For any node set the construction rule of a proximity
graph defines the corresponding edge set.
%~ The edge-set for any realization of node-placements is well defined by
%~ the construction rules of the
We measure the critical temperatures for a range of $\sigma$ and find
a power-law relationship for the GG between the critcal temperature and
the average degree, for the RNG we cannot observe a power law.
Additionally we verify that the universality class does not change by
varying $\sigma$.
%~
%~ closely related to the Delaunay triangulation and
%~ provides an interpolation between the square lattice IFM
%~ and spins distributed randomly in the plane.
%~ For a range of this interpolation the critcal temperatures are determined
%~ and it is confirmed that configurations from this range belong to the
%~ IFM universality class.

In Sec.~\ref{sec:model} proximity graphs and the Ising model
are explained. Sec.~\ref{sec:results} shows the
details and results of our simulations. Sec.~\ref{sec:conclusion}
closes this article with the conclusion and an outlook.
