The Ising model for a ferromagnet \cite{Ising1925} is perhaps the
most extensive studied model in statistical mechanics. It consists
of spins which can point up or down and interact with neighboring
spins. It shows a second order phase transition separating a
ferromagnetic from a paramagnetic phase. While there exist analytic
solutions\cite{Onsager1944,Wannier1945} describing the phase
transition for Ising models on regular lattices in two dimensions,
higher dimensional lattices and non-regular lattices are primarily studied
by Monte Carlo simulations.

Some work \cite{Janke1994,Lima2000,Lima2008} was done to determine
the critcal temperatures and confirm the universality of two-dimensional IFM
where the spins are arranged randomly in the unit square by a
Poisson point process and neighbor relationships are determined by a
Delaunay triangulation that maintains the planarity of the system.
This article extends the former research to other
irregular lattices, namely the Relative Neighborhood graph (RNG) and
Gabriel graph (GG), which belong together with the Delaunay triangulation,
to the family of \emph{proximity graphs}. The purpose of proximity graphs
is to connect nodes which are close to each other, hence they are suited
to generalize problems defined on regular lattices with nearest neighbor
relationships.
This family of graphs have been applied in geographic variation studies in
biology \cite{Sokal1978,Sokal1980,Selander1975} and as potential candidates for ``virtual backbones'' in ad-hoc
networks \cite{Kuhn2003,Bose2001,Santi2005,Karp2000,jennings2002topology}. % also machine learning

Beginning with a square lattice the nodes can be shifted randomly governed
by a tunable parameter $\sigma$ until the distribution of the nodes is
indistinguishable from a Poisson point process.
For every realization of nodes, the construction rules for the proximity
graph at hand define an edge set.
%~ The edge-set for any realization of node-placements is well defined by
%~ the construction rules of the
We measure the critical temperatures for a range of $\sigma$ and confirm that
the universality class does not change in the process.

%~
%~ closely related to the Delaunay triangulation and
%~ provides an interpolation between the square lattice IFM
%~ and spins distributed randomly in the plane.
%~ For a range of this interpolation the critcal temperatures are determined
%~ and it is confirmed that configurations from this range belong to the
%~ IFM universality class.

In Sec.~\ref{sec:model} proximity graphs and the Ising model
are explained. Sec.~\ref{sec:results} will show the
details and results of our simulations. Sec.~\ref{sec:conclusion}
closes this article with the conclusion and an outlook.
