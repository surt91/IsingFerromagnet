%~ Hallo Welt \cite{Ising1925}
%~ \blindtext
%~ ising wichtig, drosophila und so
%~ Janke und Lima haben das auch gemacht
%~ TODO: mehr finden

[todo: geo und bio cites] \cite{Karp2000} \cite{Santi2005} \cite{Bose2001} \cite{Kuhn2003} \cite{Selander1975} \cite{Sokal1978} \cite{Sokal1980}

The Ising model for a ferromagnet \cite{Ising1925} is
perhaps the most extensive studied model in statistical mechanics.
It consists of spins wich can point up or down and interact with
neighboring spins. It shows a second order phase transition between
a ferromagnetic, i.e.~all spins having the same value, and
paramagnetic phase. While there exist analytic solutions\cite{Onsager1944,Wannier1945}
describing the phase transition for Ising models on
regular lattices in two dimensions, higher dimensional lattices or
non-regular lattices are the domain of Monte Carlo simulations.

Some work \cite{Janke1994,Lima2000,Lima2008} was done to determine
the critcal temperatures and confirm the universality of 2D IFM
where the spins are arranged randomly in the unit square by a
Poisson point process and neighbor relationships are determined by a
Delaunay triangulation to maintain the planarity of the system.
This article extends the former research to other
irregular lattice types closely related to the Delaunay triangulation and
provides an interpolation between the square lattice IFM
and spins distributed randomly in the plane.
For a range of this interpolation the critcal temperatures are determined
and it is confirmed that configurations from this range belong to the
IFM universality class.

In section \ref{ssec:model} the method to distribute the sites of the
irregular lattice and in \ref{ssec:graphtypes} the rules to establish the
bonds between them is explained, section \ref{sec:results} will show the
details and results of our simulations, before section \ref{sec:conclusion}
closes this article with the conclusions and an outlook.
