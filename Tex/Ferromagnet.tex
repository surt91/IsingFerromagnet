\documentclass[a4paper,headsepline,bibtotocnumbered,12pt,titlepage,twoside]{scrartcl} % scrartcl
\usepackage[english]{babel}
\usepackage{graphicx}
\usepackage[FIGTOPCAP,tight,raggedright,nooneline]{subfigure}
\usepackage{upgreek}
\usepackage{float}
\usepackage{units}
\usepackage{url}
\usepackage[colorlinks=false, pdfborder={0 0 0}]{hyperref}
\usepackage{amsmath}
\usepackage{longtable}
%~ \usepackage{here}
\usepackage[automark]{scrpage2}
\pagestyle{scrheadings}
\clearscrheadfoot
\ohead{\pagemark}
\ihead{\headmark}
\usepackage{setspace}
\linespread{1.25}
\usepackage{geometry}
\geometry{a4paper, top=21mm, left=30mm, right=30mm, bottom=21mm,
          headsep=10mm, footskip=12mm}
\usepackage[utf8]{inputenc}
\addtokomafont{sectioning}{\rmfamily}
%\renewcommand{\figurename}{Abb.}
\usepackage[bf]{caption}
\captionsetup{format=plain}
%\renewcommand*{\captionfont}{\itshape}
\setlength{\parindent}{0pt}
%\renewcommand{\thefigure}{\arabic{section}.\arabic{figure}}
%\makeatletter \@addtoreset{figure}{section} \makeatother
%\renewcommand{\theequation}{\arabic{section}.\arabic{equation}}
%\makeatletter \@addtoreset{equation}{section} \makeatother
%\sloppy
\fussy

\usepackage{amsfonts}
\newcommand{\change}[1]{{#1}}
%\usepackage[ngerman,num]{isodate}

%\usepackage{natbib}

%%%%%%%%%%%%%%%%%%%%%%%%%%%%%%%%%%%%%%%%%%%%
%~ \documentclass[a4paper,12pt,titlepage,twoside]{scrartcl}
\usepackage[T1]{fontenc}
%~ \usepackage{lmodern}
%~ \usepackage[utf8]{inputenc}
%~ \usepackage[ngerman]{babel}
\usepackage{ae}
%~ \usepackage{amsmath}
%~ \usepackage{amssymb}
%~ \usepackage{graphicx}
%~ \usepackage{enumitem}
%~ \usepackage{caption}
%~ \usepackage{subcaption}
%~ \usepackage{float}
%~ \usepackage{wrapfig}
%~ \usepackage{url}

\usepackage{tikz}
\usetikzlibrary{patterns}

\usepackage{booktabs}

%~ \captionsetup{figurename=Abb. ,tablename=Tab. ,format=hang}
%~ \captionsetup{figurename=Abb. ,tablename=Tab., format=plain}

%für Anführungszeichen unten und oben
\newcommand{\emphgqq}[1]%
{\emph{{\glqq}#1{\grqq}}}
\newcommand{\gqq}[1]%
{{\glqq}#1{\grqq}}
\newcommand{\eqq}[1]%
{{\textquotedblleft}#1{\textquotedblright}}

%\abs{Ausdruck} %Betragsstriche, die skalieren - abgekürzt
\newcommand{\abs}[1]{\ensuremath{\left\vert#1\right\vert}}
% und das gleiche füur große Klammern
\newcommand{\brac}[1]{\ensuremath{\left(#1\right)}}
% Erwartungswert skalierend
\newcommand{\avg}[1]{\left< #1 \right>}
% ein nicht kursives d für Ableitungen/Integrale, mit etwas Platz davor, um sich etwas abzusetzten
\newcommand{\de}{\ensuremath{\,\mathrm{d}}}
% Für Einheiten: schreibt sie nicht kursiv und lässt etwas Platz zur Zahl vorher
\newcommand{\eh}[1]{\ensuremath{\,\mathrm{#1}}}
% einfaches Gradzeichen
\newcommand{\gr}{\ensuremath{^{\circ}}}
% Fehlerfortpfanzung
% dy/dz * delta z
\newcommand{\fehler}[2]%
{\ensuremath{\abs{\frac{\partial #1}{\partial #2}}\cdot \Delta #2}}

\hyphenation{}
\setcounter{lofdepth}{2}

%~ \title{Ferromagnet auf ad-hoc Netzwerk}
%~ \author{Hendrik Schawe}
%~ \date{}
%%%%%%%%%%%%%%%%%%%%%%%%%%%%%%%%%%%%%%%%%%%
\begin{document}
    \begin{titlepage}
\begin{center}

\includegraphics[width=90mm]{images/Uniol_1c.pdf}\\
\vspace{25mm}

{\LARGE  Bachelorarbeit im Fach Physik\\}

\vspace{10mm}

{\LARGE \bfseries
???
\\}

\vspace{20mm}

\begin{minipage}{0.4\textwidth}
\begin{center} \large
\emph{von}\\
\large
Hendrik Schawe\\
\end{center}
\end{minipage}

\vspace{45mm}

{\large
Betreuender Gutachter:\\
Prof.\ Dr.\ Alexander K. Hartmann\\
}
\vspace{8mm}
{\large
Zweitgutachter:\\
Dr.\ Oliver Melchert\\
}

\vfill
{\large Oldenburg, den 05.\ Juni 2013}
\newpage

\end{center}

\end{titlepage}

    \thispagestyle{empty}
    \cleardoublepage
    \setcounter{page}{1}
    \pagenumbering{roman}
    \tableofcontents
    \newpage
    \listoffigures
    \newpage
    \setcounter{page}{1}
    \pagenumbering{arabic}

    \newpage
    \section{Introduction}
        The computing capabilities of modern computers enable researchers to
collect and analyse vast amounts of data.
Further statistical systems, for which exist no analytical soultions
or only for very simplified or special cases, can be simulated.
For example to examine phasetransitions, which are defined by the abrupt
change of an observable i.e. the change of the density of water near
boiling or the change of the magnetisation of a ferromagnet near the
Curie temperature. One of the simplest models with a second order
phasetransition is the Ising model \cite{Ising1925} which is a simple
model of a ferromagnet and will be explained in more detail in section
\ref{ssec:isingmodel}. It is analytically solved for two dimensions on
some regular lattices \cite{Onsager1944} \cite{Wannier1945}.
In this thesis it's behavior near the Curie temperature -- also called
critical temperature -- on some irregular lattices corresponding to
proximity graphs (see section \ref{ssec:graphtypes}) will be examined
using the Monte Carlo simulations described in section \ref{sec:montecarlo}.

Unfortunatly the memory of any computer is small in comparison with the
thermodynamic limit. This leads to \emph{finite size effects}.
In section \ref{ssec:finitesize} will be discussed how to manage them.\\

Note that in the scope of this thesis the Boltzmann constant \(k_{B}=1\)
for the sake of simplicity.


    \section{Monte Carlo Simulations}
        \label{sec:montecarlo}
Monte Carlo simulations are (probably) named after the casino in Monaco
\cite{NewmanBarkema1999} because both rely on randomness.\\
The idea behind Monte Carlo simulations is to take random samples of
the observable, which should be measured, and estimate the observable from
this samples. In example the Monte Carlo integration chooses random
points within the bounding box of the fuction and estimates the integral
as the fraction of points below the function and points above muliplied
by the area of the bounding box.
Of course one wants to sample more points, where the area below the
function is big. This can be achieved by \emph{Importance Sampling}. One
generates random numbers according to some probabilty distribution and
weights the result accordingly.\\
%~ To improve this method: \emph{Importance Sampling}
%~ means that one takes more samples where the contribution of the
%~ function to the integral is bigger by using random numbers
%~ distributed........
The same principle is used to sample properties of statistical systems
by generating random states.
But it is difficult to create random state of the Ising system in
equilibrium, hence one uses \emph{Markov Chains} to generate new states
\(\nu\) from old ones \(\mu\).
It is important that the transition probabilities \(A(\mu \to \nu)\)
obey \emph{Detailed Balance} and \emph{Ergodicity}.
\emph{Detailed Balance} means  that the probability to leave a state is
the same as the probability to enter the state in equilibrium
\(p_\mu A(\mu \to \nu) = p_\nu A(\nu \to \mu)\) with \(p_\mu\) the
probability to be in state \(\mu\).
And \emph{Ergodicity} requires that every possible state is reachable
from every other state in finite time. \cite{NewmanBarkema1999} \cite{Katzgraber2011}
Otherwise the samples might not be representative for the whole system.\\

\subsection{Monte Carlo Algorithms}
    \paragraph{Metropolis}
        A Metropolis Monte Carlo simulation of an Ising model will
        choose a random spin, calculate the energy change
        \(\Delta \hat H\) defined in eq. \eqref{eq:dH} that would result
        from a flip of and excute the flip with the probability \(A\)
        eq. \eqref{eq:metropolis} \cite{NewmanBarkema1999} \cite{Katzgraber2011}.
        \begin{equation}
            \Delta \hat H = \hat H(\nu) - \hat H(\mu)\\
            \label{eq:dH}
        \end{equation}
        \begin{equation}
            A(\mu \to \nu)
            \begin{cases}
                1                                 & \Delta \hat H \le 0 \\
                \exp{\brac{-\beta \Delta \hat H}} & \Delta \hat H > 0
            \end{cases}
            \label{eq:metropolis}
        \end{equation}
        So if a transition lowers the energy it is always done. This
        results in a good ratio between calculations and transitions.

    \paragraph{Wolff}
        Close to the critical temperature \(T_c\) the Metropolis
        gets slower. This is called \emph{critical slowing} down and the
        cause is beyond the scope of this thesis.\\
        Using a cluster algorithm like the Wolff
        algorithm \cite{Wolff1989} speeds things up.
        For an Ising model the Wolff algorithm builds a cluster of sites
        starting with a random site and adding neighboring sites of the
        same spin with probability
        \(P_{\mathrm{add}} = 1-\exp\brac{-2\beta J}\)
        where \(J\) is the coupling constant (explained in section
        \ref{ssec:isingmodel}). The neighboring sites of the added sites
        are also considered and so forth. When there are no more sites
        to add, the spin of every site in the cluster is flipped
        \cite[S. ??]{NewmanBarkema1999} \cite[S. 151f]{Katzgraber2011}.
        This leads fast to new uncorrelated states at the critical
        temperature because big clusters are flipped. But there are not
        much advantages at high or low temperatures. At low temperatures
        the cluster will consist of almost all sites such that all but
        very few spins will be flipped. At high temperatures the cluster
        will only contain very few sites.
        Both situaitions have no advantage against the Metropolis algorithm.

    \paragraph{Parallel Tempering}
        %~ The main aim is to obtain the critical temperatures
        %~ \(T_c\) for different disorder paramters \(\sigma\).
        %~ Therefore it is necessary to simulate for many temperatures,
        %~ so that \emph{Parallel Tempering}\footnote{Before R. H.
            %~ Swendsen published this paper, a algorithm \(MC^3\) was
            %~ already published with the same idea. [citation needed]}
        %~ \cite{ParallelTempering1986} is a suited algorithm.
        Parallel Tempering simulates many identical systems at different
        temperatures and periodically swaps the spin configurations
        between two neighboring temperatures with probability \(P\) from
        eq. \eqref{eq:partemp} \cite[S. ??]{NewmanBarkema1999} \cite[S. 155ff]{Katzgraber2011}.
        \begin{equation}
            P((E_i,T_i) \to (E_{i+1},T_{i+1})) = \min\brac{1,\exp\brac{\brac{E_{i+1}-E_i}\brac{\frac{1}{T_{i+1}}-\frac{1}{T_i}}}}
            \label{eq:partemp}
        \end{equation}
        This has the advantages that correlation times of single
        temperatures are far smaller because their spin configuration
        gets often replaced by an other uncorrelated configuration. In
        many cases the more important advantage is, that a system which
        is trapped in a local minimum at a given temperature, can travel
        to higher temperatures, leave it's local minimum and cool down
        again in a lower minimum.

    \paragraph{Implementation Details}
        Here a mixture of the above three algorithms is used.
        Each sweep \(N\) Metropolis spin flips, one Wolff cluster flip
        and one Parallel Tempering swap are performed. Where \(N\) is the
        count of sites.
%~ Monte Carlo Simulationen:\\
    %~ Wolff-Cluster Algorithmus \cite{Wolff1989} (siehe auch \cite[S. xx]{NewmanBarkema1999}),
    %~ Metropolis Sweep \cite{Metropolis1953} (siehe auch \cite[S. xx]{NewmanBarkema1999}),
    %~ Parallel Tempering \cite{ParallelTempering1986} (siehe auch \cite[S. xx]{NewmanBarkema1999} \cite[S. xx]{Katzgraber2011})\\

\subsection{Equilibration- and Autocorrelation Time}
\label{ssec:eqtime}
    To generate an equilibrium state one starts with an arbitrary state
    and waits until it is equilibrated. The count of sweeps till
    equilibrium is called \emph{equilibration time} \(t_{eq}\).
    All measurement should start after this time.\\
    To determine when the system is in equilibrium, one can watch the
    development of some observables and take the point at which there
    are no big changes anymore as equilibrium. In fig.
    \ref{fig:equiandauto}\subref{sfig:equiandauto:equiE}
    the equilibrium is reached after approximately \(N=50\) sweeps for
    either an initial condition of all spins up and all spins random. It
    does not harm to double that value to be save.
    \begin{figure}[htbp]
        \centering
        \subfigure[Example of an equilibrating Ising system][]{
                \label{sfig:equiandauto:equiE}
                \includegraphics[width=0.47\textwidth]{plots/equiE}
        }
        \subfigure[Example of the autocorrelation of an Ising system][]{
                \label{sfig:equiandauto:autoM}
                \includegraphics[width=0.47\textwidth]{plots/autoM}
        }
        \caption[Examples for equilibration and autocorrelation]
                {\subref{sfig:equiandauto:equiE} Example of a Ising
                    system reaching equilibrium at \(T=2.36\) and
                 \subref{sfig:equiandauto:autoM} the
                    autocorrelation of an Ising system at \(T=2.40\)
                    (only Metropolis sweeps).
                }
        \label{fig:equiandauto}
    \end{figure}

    Because every state is generated from the state before, measurements
    of subsequent states are correlated. To determine when two states
    are independet, one calculates the normalized autocorrelation function
    \(\frac{\chi(t)}{\chi(0)}\), which should decay exponentially
    \(\chi(t) \propto \exp(t/\tau)\). This is visible in the half
    logarithmic plot \ref{fig:equiandauto}\subref{sfig:equiandauto:autoM}.
    To get the autocorrelation time one can integrate \(\tau = \int \frac{\chi(t)}{\chi(0)} \de t\).
    \(\tau\) is an estimate after which time two samples are not
    correlated anymore. \cite[S. ??]{NewmanBarkema1999} \cite[S. 150f]{Katzgraber2011}.
    To make sure that the error is not underestimated one should wait
    \(2\tau\) sweeps between two measurements.



    \section{Model and Impelementation}
        \subsection{The Disordered Ising Model}
\label{ssec:isingmodel}
    The examined model is a modified 2D Ising model.
    The most common definition of the Ising model, to which will be referred
    as the \emph{standard Ising model}, is a square lattice with edge length \(L\) and
    \(N=L^2\) sites. Each site has a magnetic moment, the spin. Each
    spin can take a value \(s \in \{-1,+1\}\) and interacts with its
    nearest neighbors described by the Hamiltonian from eq. \eqref{eq:hamiltonian}
    \begin{equation}
        H = - \sum_{\avg{i,j}}J_{ij}s_{i}s_{j} - \sum_i \tilde{H}_i s_i
        \label{eq:hamiltonian}
    \end{equation}
    \(\avg{i,j}\) refers to nodes \(i\) and \(j\) which are nearest
    neighbors. And \(J_{ij}\) is the coupling constant between \(i\) and
    \(j\). If \(J_{ij} > 0 \ \forall i,j\) the model resembles a ferromagnet.
    \(\tilde{H}_i\) denotes the outer magnetic field at the position of
    site \(i\).\\
    In this thesis \(\tilde{H}_i=0 \ \forall i\). The most important
    modification is, that the sites of the square lattice are displaced.
    The displacement is randomly Gauß distributed with the standard
    deviation \(\sigma\), i.e. the \(x\) and \(y\) coordinates of the
    sites are displaced by random \(\Delta x\) and \(\Delta y\) drawn
    from a Gauß distribution eq. \eqref{eq:gauss}.
    This is sketched in fig. \ref{fig:displacement}.
    \begin{equation}
        f(x)=\frac{1}{\sqrt{2\pi}\sigma}\mathrm{e}^{-\frac{x^2}{2\sigma^2}}
        \label{eq:gauss}
    \end{equation}
    \[x \to x + \Delta x\]
    \[y \to y + \Delta y\]
    \begin{figure}[htbp]
        \centering
        \begin{tikzpicture}[scale=1.5, declare function={
        normal(\x,\m,\y) = 1/2/exp((\x-\m)*(\x-\m)/2/(\s^2))-\y;
      }]
    \def\s{0.5}

    \draw[dotted] (-2,0) -- (4,0);
    \draw[dotted] (0,-2) -- (0,2);
    \draw[dotted] (2,-2) -- (2,2);

    % Draw and label normal distribution function
    \def\dxa{0.4}
    \def\dya{-0.8}

    \draw (0+\dxa,0+\dya) -- node [right] {$\Delta x$} (0+\dxa, {normal(\dxa,0,0)});
    \draw (0+\dxa,0+\dya) -- node [below] {$\Delta y$} ({-normal(\dya,0,0)}, 0+\dya);
    \draw[->] (0,0) -- (0+\dxa*0.9,0+\dya*0.9);

    \draw[color=black,domain=-1.5:1.5] plot [smooth] (\x,{normal(\x,0,0)}) node[right] {};
    \fill (0, 0) circle(0.08);
    \draw[color=black] (0+\dxa, 0+\dya) circle(0.08);
    \draw[color=black,domain=-1.5:1.5,rotate=90] plot [smooth] (\x,{normal(\x,0,0)}) node[right] {};


    \def\dxb{0.5}
    \def\dyb{0.2}

    \draw[color=red] (2+\dxb,0+\dyb) -- node [right] {$\Delta x_{2}$} (2+\dxb, {normal(\dxb,0,0)});
    \draw[color=red] (2+\dxb,0+\dyb) -- node [below] {$\Delta y_{2}$} ({2-normal(\dyb,0,0)}, 0+\dyb);
    \draw[color=red,->] (2,0) -- (2+\dxb*0.9,0+\dyb*0.9);

    \draw[color=red,domain=0.5:3.5] plot [smooth] (\x,{normal(\x,2,0)}) node[right] {};
    \fill[color=red] (2, 0) circle(0.08);
    \draw[color=red] (2+\dxb, 0+\dyb) circle(0.08);
    \draw[color=red,domain=-1.5:1.5,rotate=90] plot [smooth] (\x,{normal(\x,0,2)}) node[right] {};
\end{tikzpicture}

        \caption[Sketch how the Displacement Works]
        {
            Sketch how the displacement of the nodes works. The nodes
            get displaced by \(\Delta x\) and \(\Delta y\) drawn from the
            distributions displayed next to the points. The original
            square lattice is indicated by dashed lines.
        }
        \label{fig:displacement}
    \end{figure}\\
    This \(\sigma\) is also called \emph{disorder parameter} in the following.
    Because most sites will only have one nearest neighbor after the
    displacement, the lattice would collapse to many very small clusters.
    To avoid this, the new "nearest" neighbors are those sites connected
    by an edge. The edges are constructed according to
    one of the two in section \ref{ssec:graphtypes} defined rules,
    so that the lattice represents a proximity graph. Note that edges
    of a proximity graph do not cross each other, hence the 2D character
    of the lattice is preserved. The coupling constant \(J\) gets
    identified with edge weights. The weight of an edge \(E_{ij}\) is
    \(J_{ij} = \exp (\alpha (1-d_{ij}))\) where \(d_{ij}\) is the Euclidean
    distance between the nodes \(i\) and \(j\). The free parameter
    \(\alpha\) is set to \(\alpha = 0.5\) inspired by \cite{Lima2000}.
    The boundary is periodic e.g. nodes near the right edge can be
    connected to nodes near the left edge and vice versa. Analogous the
    top and bottom edges are connected. One can imagine that the model
    lives on the surface of a torus as pictured in fig. \ref{fig:torusRNG}.
    In subsequent graphics, the graphs will be unwrapped to rectangular
    shapes. Connections which cross a periodic boundary are indicated
    by edges which connect to an dashed node.
    \begin{figure}[htbp]
        \centering
        \includegraphics[width=0.45\textwidth]{images/torus}
        \caption[A Graph on a Torus to Visualise Periodic Boundary Conditions]
        {
            A graph on a torus to visualize periodic boundary conditions.
            Note that the lattice on this torus is not a square but has
            a height to width ratio of 1:4. At 1:1 the torus would cut
            itself. Hence, the torus represents the geometry of the model
            not perfectly, but gives very quick the right idea.
            %Also the shades are of course only a guide to the eye.
        }
        \label{fig:torusRNG}
    \end{figure}\\
    For \(\sigma = 0\) this is the standard Ising model with \(J = 1\),
    for which exists an analytic solution \cite{Onsager1944}. And for
    \(\sigma \gtrsim 1\) the nodes are distributed randomly. This case
    is already studied for constant \(J\) on the Delaunay triangulation
    in \cite{Janke1994}.\\

\subsection{Gabriel- and Relative Neighborhood Graph}
\label{ssec:graphtypes}
    A graph \(G(V,E)\) is a set of nodes \(V\) and edges \(E\).\\
    All here mentioned graph types are \emph{proximity graphs}. They are
    connecting nodes which are by some metric near to each other.
    Hence they are suited to generalize problems defined on regular
    lattices with nearest neighbor relationships, like the Ising model
    from the section \ref{ssec:isingmodel}.
    In this thesis the distance is always determined by the Euclidean
    metric in two dimensions, though in principle every metric in any
    dimension can be used.\\

    The Gabriel graph (GG) \cite{Gabriel1969} is a subgraph of the
    Delaunay triangulation. Two nodes \(i\) and \(j\) with distance
    \(d_{ij}\) are connected with an edge, if a circle with its
    center on half way between \(i\) and \(j\) and radius
    \(r = \frac d 2\) contains no other nodes. This area will be
    called \emph{lune} in the following. See also Figure
    \ref{fig:lunes}\subref{sfig:lunes:def}.\\
    The Relative Neighborhood graph (RNG) \cite{Toussaint1980} is a
    subgraph of the GG. Two nodes \(i\) and \(j\) with
    distance \(d_{ij}\) are connected, if no other node is in the
    \emph{lune}. The lune is defined as the intersection of two
    circles with radius \(r = d\) and centers on \(i\) and \(j\).
    See also Figure \ref{fig:lunes}\subref{sfig:lunes:def}.
    \begin{figure}[htbp]
        \centering
        \subfigure[Definition of the Lunes][]{
            \label{sfig:lunes:def}
            \documentclass{standalone}
\usepackage{tikz}
\usetikzlibrary{patterns}

\begin{document}
    \tikzset{
        hatch distance/.store in=\hatchdistance,
        hatch distance=10pt,
        hatch thickness/.store in=\hatchthickness,
        hatch thickness=2pt
    }

    \makeatletter
    \pgfdeclarepatternformonly[\hatchdistance,\hatchthickness]{flexible hatch no}
    {\pgfqpoint{0pt}{0pt}}
    {\pgfqpoint{\hatchdistance}{\hatchdistance}}
    {\pgfpoint{\hatchdistance-1pt}{\hatchdistance-1pt}}%
    {
        \pgfsetcolor{\tikz@pattern@color}
        \pgfsetlinewidth{\hatchthickness}
        \pgfpathmoveto{\pgfqpoint{0pt}{0pt}}
        \pgfpathlineto{\pgfqpoint{\hatchdistance}{\hatchdistance}}
        \pgfusepath{stroke}
    }
    \makeatletter
    \pgfdeclarepatternformonly[\hatchdistance,\hatchthickness]{flexible hatch nw}
    {\pgfqpoint{0pt}{0pt}}
    {\pgfqpoint{\hatchdistance}{\hatchdistance}}
    {\pgfpoint{\hatchdistance-1pt}{\hatchdistance-1pt}}%
    {
        \pgfsetcolor{\tikz@pattern@color}
        \pgfsetlinewidth{\hatchthickness}
        \pgfpathmoveto{\pgfqpoint{0pt}{\hatchdistance}}
        \pgfpathlineto{\pgfqpoint{\hatchdistance}{0pt}}
        \pgfusepath{stroke}
    }

    \begin{tikzpicture}
        \clip (-2,2.25) rectangle (2,-1.75);

        \begin{scope}
            \clip (-1, 0.5) circle(2.06155281281);
            %~ \fill[fill=blue!20] (1, 0) circle(2.06155281281);
            %~ \draw[pattern=north west lines] (1, 0) circle(2.06155281281);
            \draw[pattern=flexible hatch no,hatch distance=10pt,hatch thickness=0.7pt] (1, 0) circle(2.06155281281);
        \end{scope}

        %~ \fill[fill=white] (0, 0.25) circle(1.0307764064);
        %~ \draw[pattern=north east lines] (0, 0.25) circle(1.0307764064);
        \draw[pattern=flexible hatch nw,hatch distance=10pt,hatch thickness=0.7pt] (0, 0.25) circle(1.0307764064);
        \draw[thick] (0, 0.25) circle(1.0307764064);

        \draw[thick] (-1, 0.5) circle(2.06155281281);
        \fill (-1, 0.5) circle(0.1);
        \draw[thick] (1, 0) circle(2.06155281281);
        \fill (1, 0) circle(0.1);
        \draw[thick] (1, 0) -- (-1, 0.5);
    \end{tikzpicture}
\end{document}

        }
        \subfigure[RNG example][]{
            \label{sfig:lunes:rng}
            \includegraphics[width=0.3\textwidth]{images/RNG/L12S03.pdf}
        }
        \subfigure[GG example][]{
            \label{sfig:lunes:gg}
            \includegraphics[width=0.3\textwidth]{images/GG/L12S03.pdf}
        }
        \caption[Gabriel - and Relative Neighborhood Graph]
        {
            \subref{sfig:lunes:def} Lunes of RNG (hatched region) and
                GG (cross hatched region)
            \subref{sfig:lunes:rng} Example of a RNG on periodic
                boundary conditions. Periodic nodes are dashed.
            \subref{sfig:lunes:gg} Example of a GG on
                periodic boundary conditions. Periodic nodes are dashed.
        }
        \label{fig:lunes}
    \end{figure}\\
    To construct these graphs the simple way is to test for each
    pair of nodes if any other node lies in
    the lune of the pair. That is of complexity \(O (N^3)\), because
    there are \(N(N-1)\) pairs and for each \(N-2\) nodes to test. So
    the product is of order \(O(N^3)\)\\
    To reduce the complexity one can first create a Delaunay
    Triangulation in complexity \(O (N \log N)\)
    \cite{RNGCell} and test the criterion for each edge, because
    the Delaunay triangulation is a supergraph of both. But the
    implementation of a Delaunay triangulation algorithm is not trivial
    and the generation of the graphs is not time critical in the scope
    of this bachelor thesis.\\
    So a trade off is to use basically the simple method but only test
    the criterion for nodes which are near to the lune and abort if
    one node inside the lune is found. To determine which nodes are
    near the lune one can subdivide the area in \emph{cells} and save
    for each cell a list with nodes lying inside it like presented in
    \cite{RNGCell}.
    Now it is just necessary to test the nodes in the cells which
    resemble a rectangular bounding box of the lune. Most pairs will be
    far away from each other and the cells in the middle of the bounding
    box are completely inside the lune so that only one node has to be
    tested to discard an edge between them. Connected nodes are near to
    each other so that only very few cells have to be tested.\\
    Indeed this method reduced the time needed to construct a RNG with
    \(N=32^2\) and \(N=64^2\) by a factor of
    over \(15\) respectively \(40\). Though the complexity is still of
    order \(O(N^2)\) in the best case, because for every pair at least
    one check has to be performed.


    \section{Results}
        \subsection{Critical Temperature}
    The evaluation of the Binder cumulant's intersections, yields the
    critical temperatures \(T_c\), which are normalized and plotted in
    fig. \ref{fig:Tc}.
    \begin{figure}[htbp]
        \centering
        \subfigure[][]
        {
            \label{sfig:Tc:RNG}
            \includegraphics[width=0.45\textwidth]{plots/RNG_Tc_norm}
        }
        \subfigure[][]
        {
            \label{sfig:Tc:GG}
            \includegraphics[width=0.45\textwidth]{plots/GG_Tc_norm}
        }
        \caption[Critical Temperature over different disturbance parameters]
                {Normalized critical temperatures over different
                 disturbance parameters for
                 \subref{sfig:Tc:RNG} the Relative Neighborhood Graph and
                 \subref{sfig:Tc:GG} the Gabriel Graph.
                }
        \label{fig:Tc}
    \end{figure}
    % I don't know what all that means...

\subsection{Critical Exponents}
    For \(\sigma \in \{0,0.1,0.5\}\) a finite size scaling analysis was
    performed to determine the critical exponents \(\beta, \gamma, \nu\)
    using \texttt{autoscale.py} \cite{autoscale2009}. The values for
    \(\sigma = 0\) are analytically known \cite{Pelissetto2002}. The
    values for all other \(\sigma\) should be the same according to ???
    [citiation needed]. Like in tab. \ref{tab:critExp} to see, most values
    are matching the expectations. Most \(\beta\) seem to be a bit too
    big, but they are close enough to the expectations to be explained
    by the fact that small systems (\(L=32,64\)) were used for the
    analysis. [citation needed]
    \begin{table}[htbp]
        \center
        \begin{tabular}{l l l l l}
            \toprule
             & \multicolumn{1}{c}{\(\sigma\)} & \multicolumn{1}{c}{\(\nu\)} & \multicolumn{1}{c}{\(\gamma\)} & \multicolumn{1}{c}{\(\beta\)}\\
            \midrule
            exact (\cite[p. 59]{Pelissetto2002}) & \multicolumn{1}{c}{\(0\)} & \multicolumn{1}{c}{\(1\)} & \multicolumn{1}{c}{\(-\frac{7}{4}\)} & \multicolumn{1}{c}{\(\frac{1}{8}\)}\\
            \midrule
            Gabriel      & 0.0 & 1.008(4) & -1.735(2) & 0.1262(4)\\
                         & 0.1 & 1.02(1)  & -1.744(5) & 0.133(6) \\
                         & 0.5 & 1.009(8) & -1.75(1)  & 0.125(13)\\
            \midrule
            Relative N.  & 0.0 & 1.007(2) & -1.739(2) & 0.130(1) \\
                         & 0.1 & 0.99(1)  & -1.746(5) & 0.133(4) \\
                         & 0.5 & 1.00(2)  & -1.75(2)  & 0.143(13)\\
            \bottomrule
        \end{tabular}
        \caption{Critical exponents for different \(\sigma\)}
        \label{tab:critExp}
    \end{table}

    %~ <J>\(\sigma\), Anmerkung zur Anomalie des GG und der Ähnlichkeit von <J> zu \(T_c\)\\
%~
    %~ Erklärung zu Autokorreltationszeit \(\tau\) und Equilibrierungszeit \(t_eq\)\\
    %~ Darstellung von Suszeptibilität \(\chi\), spezifischer Wärme \(c\), mittlerer Magnetisierung \(<m>\) über Unordnungsparamter \(\sigma\)\\
    %~ Bestimmung der kritischen Punkte\\
        %~ Polyfit 4ten Grades durch Binder Kumulante \cite{Binder1981}\\
        %~ Finite Size Scaling Vergleich der Exponenten (AutoScale \cite{Melchert2009}, Vergleich \cite[S. 59]{Pelissetto2002})\\



    \section{Zusammenfassung}
        %~ \input{4_Zusammenfassung}

    \bibliography{lit}
    %~ \bibliographystyle{alpha}
    \bibliographystyle{amsplain}

\end{document}
