\documentclass[a4paper,12pt,titlepage,twoside]{scrartcl}
\usepackage[T1]{fontenc}
\usepackage{lmodern}
\usepackage[utf8]{inputenc}
\usepackage[ngerman]{babel}
\usepackage{ae}
\usepackage{amsmath}
\usepackage{amssymb}
\usepackage{graphicx}
\usepackage{enumitem}
\usepackage{caption}
\usepackage{subcaption}
\usepackage{float}
\usepackage{wrapfig}
\usepackage{url}

\usepackage{booktabs}

%~ \captionsetup{figurename=Abb. ,tablename=Tab. ,format=hang}
\captionsetup{figurename=Abb. ,tablename=Tab., format=plain}

%für Anführungszeichen unten und oben
\newcommand{\emphgqq}[1]%
{\emph{{\glqq}#1{\grqq}}}
\newcommand{\gqq}[1]%
{{\glqq}#1{\grqq}}
\newcommand{\eqq}[1]%
{{\textquotedblleft}#1{\textquotedblright}}

%\abs{Ausdruck} %Betragsstriche, die skalieren - abgekürzt
\newcommand{\abs}[1]{\ensuremath{\left\vert#1\right\vert}}
% und das gleiche füur große Klammern
\newcommand{\brac}[1]{\ensuremath{\left(#1\right)}}
% ein nicht kursives d für Ableitungen/Integrale, mit etwas Platz davor, um sich etwas abzusetzten
\newcommand{\de}{\ensuremath{\,\mathrm{d}}}
% Für Einheiten: schreibt sie nicht kursiv und lässt etwas Platz zur Zahl vorher
\newcommand{\eh}[1]{\ensuremath{\,\mathrm{#1}}}
% einfaches Gradzeichen
\newcommand{\gr}{\ensuremath{^{\circ}}}
% Fehlerfortpfanzung
% dy/dz * delta z
\newcommand{\fehler}[2]%
{\ensuremath{\abs{\frac{\partial #1}{\partial #2}}\cdot \Delta #2}}

\hyphenation{}

\title{Ferromagnet auf ad-hoc Netzwerk}
\author{Hendrik Schawe}
\date{}

\begin{document}
    \maketitle

    \cleardoublepage
    \tableofcontents

    \clearpage
    \section{Einleitung}
        %~ Get critical temperatures for different graphs!


    \section{Theorie}
        \subsection{Modell}
    Ising Modell \(\hat{H} = \sum_{<i,j>}J_{ij}s_{i}s_{j}\)\\
    zufälliges Gitter: Gaußverteilt\\
    Kopplungskonstanten abstandsabhängig: \(J_{ij}=e^{(1-\alpha)d_{ij}}\)\\
    Monte Carlo Simulationen:\\
        Wolff-Cluster Algorithmus \cite{Wolff1989} (siehe auch \cite[S. xx]{NewmanBarkema1999}),
        Metropolis Sweep \cite{Metropolis1953} (siehe auch \cite[S. xx]{NewmanBarkema1999}),
        Parallel Tempering \cite{ParallelTempering1986} (siehe auch \cite[S. xx]{NewmanBarkema1999} \cite[S. xx]{Katzgraber2011})\\
        Erklärung zu Autokorreltationszeit \(\tau\) und Equilibrierungszeit \(t_eq\)\\
        Darstellung von Suszeptibilität \(\chi\), spezifischer Wärme \(c\), mittlerer Magnetisierung \(<m>\) über Unordnungsparamter \(\sigma\)\\
    Relative Neighborhood Graph \cite{Toussaint1980}\\
    Gabriel Graph \cite{Gabriel1969}\\
        Cell Algorithmus ohne Delaunay Triangulation \cite{RNGCell}\\
        <J>(\sigma), Anmerkung zur Anomalie des GG und der ähnlichkeit zu \(T_c\)\\
    Bestimmung der kritischen Punkte\\
        Polyfit 4ten Grades durch Binder Kumulante \cite{Binder1981}\\
        Finite Size Scaling Vergleich der Exponenten (AutoScale \cite{Melchert2009}, Vergleich \cite[S. 59]{Pelissetto2002})\\


    \section{Auswertung}
        %~ \input{3_Auswertung}

    \section{Zusammenfassung}
        %~ \input{4_Zusammenfassung}

    \bibliography{lit}
    \bibliographystyle{alpha}
\end{document}
