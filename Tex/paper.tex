\documentclass[a4paper,headsepline,bibtotocnumbered,12pt,titlepage,twoside]{scrartcl}

\usepackage[utf8]{inputenc}
\usepackage[T1]{fontenc}

\usepackage[english]{babel}

\usepackage{graphicx}
\usepackage{upgreek}
\usepackage{float}
\usepackage{units}
\usepackage{url}

\usepackage{amsmath}
\usepackage{amssymb}
\usepackage{amsfonts}

\usepackage{longtable}

\usepackage[FIGTOPCAP,tight,raggedright,nooneline]{subfigure}
\usepackage[colorlinks=false, pdfborder={0 0 0}]{hyperref}

\usepackage[automark]{scrpage2}
\pagestyle{scrheadings}
\clearscrheadfoot
\ohead{\pagemark}
\ihead{\headmark}
\usepackage{setspace}
\linespread{1.25}
\usepackage{geometry}
%~ \geometry{a4paper, top=21mm, left=30mm, right=30mm, bottom=21mm,
          %~ headsep=10mm, footskip=12mm}
\geometry{a4paper, top=21mm, bottom=21mm, headsep=10mm, footskip=12mm}

\addtokomafont{sectioning}{\rmfamily}
\usepackage[bf]{caption}
\captionsetup{format=plain}
\setlength{\parindent}{0pt}
\fussy

\newcommand{\change}[1]{{#1}}

\usepackage{ae}
\usepackage{booktabs}

\usepackage{tikz}
\usetikzlibrary{patterns}

%\abs{Ausdruck} %Betragsstriche, die skalieren - abgekürzt
\newcommand{\abs}[1]{\ensuremath{\left\vert#1\right\vert}}
% und das gleiche füur große Klammern
\newcommand{\brac}[1]{\ensuremath{\left(#1\right)}}
% Erwartungswert skalierend
\newcommand{\avg}[1]{\left< #1 \right>}
% ein nicht kursives d für Ableitungen/Integrale, mit etwas Platz davor, um sich etwas abzusetzten
\newcommand{\de}{\ensuremath{\,\mathrm{d}}}
% Für Einheiten: schreibt sie nicht kursiv und lässt etwas Platz zur Zahl vorher
\newcommand{\eh}[1]{\ensuremath{\,\mathrm{#1}}}
% einfaches Gradzeichen
\newcommand{\gr}{\ensuremath{^{\circ}}}
% Fehlerfortpfanzung
% dy/dz * delta z
\newcommand{\fehler}[2]%
{\ensuremath{\abs{\frac{\partial #1}{\partial #2}}\cdot \Delta #2}}

\hyphenation{}
\setcounter{lofdepth}{2}

\numberwithin{equation}{section}
\numberwithin{figure}{section}
\numberwithin{table}{section}

\begin{document}
    \tableofcontents
    \newpage
    \listoffigures
    \listoftables

    \section{Abstract}
        We perform Monte Carlo simulations to determine the critical temperatures
of an Ising Ferromagnet (IFM) coupled to different types of 2D proximity
graphs.
%~ In particular we consider Relative Neighborhood graphs and Gabriel graphs
%~ which are subgraphs of the Delaunay Triangulation for a given set of points.
The graphs are derived from square lattices where nodes are displaced by
a Gaussian distributed random variable.
%~ The limit of maximal disorder thus corresponds to a Poisson point process.
The deviation of the proximity graph from a square lattice is
governed by the width \(\sigma\) of the Gaussian distribution.
In our model, the coupling strength depends on the euclidian distance
between the coupled spins.
The critical temperatures are shown to depend mainly on the average degree
and the type of the underlying proximity graph.
We further verify that the model lies within the universality class of
the 2D IFM.


    \section{Introduction}
        The computing capabilities of modern computers enable researchers to
collect and analyse vast amounts of data.
Further statistical systems, for which exist no analytical soultions
or only for very simplified or special cases, can be simulated.
For example to examine phasetransitions, which are defined by the abrupt
change of an observable i.e. the change of the density of water near
boiling or the change of the magnetisation of a ferromagnet near the
Curie temperature. One of the simplest models with a second order
phasetransition is the Ising model \cite{Ising1925} which is a simple
model of a ferromagnet and will be explained in more detail in section
\ref{ssec:isingmodel}. It is analytically solved for two dimensions on
some regular lattices \cite{Onsager1944} \cite{Wannier1945}.
In this thesis it's behavior near the Curie temperature -- also called
critical temperature -- on some irregular lattices corresponding to
proximity graphs (see section \ref{ssec:graphtypes}) will be examined
using the Monte Carlo simulations described in section \ref{sec:montecarlo}.

Unfortunatly the memory of any computer is small in comparison with the
thermodynamic limit. This leads to \emph{finite size effects}.
In section \ref{ssec:finitesize} will be discussed how to manage them.\\

Note that in the scope of this thesis the Boltzmann constant \(k_{B}=1\)
for the sake of simplicity.


    \section{Model}
        \subsection{The Disordered Ising Model}
\label{ssec:isingmodel}
    The examined model is a modified 2D Ising model.
    The most common definition of the Ising model, to which will be referred
    as the \emph{standard Ising model}, is a square lattice with edge length \(L\) and
    \(N=L^2\) sites. Each site has a magnetic moment, the spin. Each
    spin can take a value \(s \in \{-1,+1\}\) and interacts with its
    nearest neighbors described by the Hamiltonian from eq. \eqref{eq:hamiltonian}
    \begin{equation}
        H = - \sum_{\avg{i,j}}J_{ij}s_{i}s_{j} - \sum_i \tilde{H}_i s_i
        \label{eq:hamiltonian}
    \end{equation}
    \(\avg{i,j}\) refers to nodes \(i\) and \(j\) which are nearest
    neighbors. And \(J_{ij}\) is the coupling constant between \(i\) and
    \(j\). If \(J_{ij} > 0 \ \forall i,j\) the model resembles a ferromagnet.
    \(\tilde{H}_i\) denotes the outer magnetic field at the position of
    site \(i\).\\
    In this thesis \(\tilde{H}_i=0 \ \forall i\). The most important
    modification is, that the sites of the square lattice are displaced.
    The displacement is randomly Gauß distributed with the standard
    deviation \(\sigma\), i.e. the \(x\) and \(y\) coordinates of the
    sites are displaced by random \(\Delta x\) and \(\Delta y\) drawn
    from a Gauß distribution eq. \eqref{eq:gauss}.
    This is sketched in fig. \ref{fig:displacement}.
    \begin{equation}
        f(x)=\frac{1}{\sqrt{2\pi}\sigma}\mathrm{e}^{-\frac{x^2}{2\sigma^2}}
        \label{eq:gauss}
    \end{equation}
    \[x \to x + \Delta x\]
    \[y \to y + \Delta y\]
    \begin{figure}[htbp]
        \centering
        \begin{tikzpicture}[scale=1.5, declare function={
        normal(\x,\m,\y) = 1/2/exp((\x-\m)*(\x-\m)/2/(\s^2))-\y;
      }]
    \def\s{0.5}

    \draw[dotted] (-2,0) -- (4,0);
    \draw[dotted] (0,-2) -- (0,2);
    \draw[dotted] (2,-2) -- (2,2);

    % Draw and label normal distribution function
    \def\dxa{0.4}
    \def\dya{-0.8}

    \draw (0+\dxa,0+\dya) -- node [right] {$\Delta x$} (0+\dxa, {normal(\dxa,0,0)});
    \draw (0+\dxa,0+\dya) -- node [below] {$\Delta y$} ({-normal(\dya,0,0)}, 0+\dya);
    \draw[->] (0,0) -- (0+\dxa*0.9,0+\dya*0.9);

    \draw[color=black,domain=-1.5:1.5] plot [smooth] (\x,{normal(\x,0,0)}) node[right] {};
    \fill (0, 0) circle(0.08);
    \draw[color=black] (0+\dxa, 0+\dya) circle(0.08);
    \draw[color=black,domain=-1.5:1.5,rotate=90] plot [smooth] (\x,{normal(\x,0,0)}) node[right] {};


    \def\dxb{0.5}
    \def\dyb{0.2}

    \draw[color=red] (2+\dxb,0+\dyb) -- node [right] {$\Delta x_{2}$} (2+\dxb, {normal(\dxb,0,0)});
    \draw[color=red] (2+\dxb,0+\dyb) -- node [below] {$\Delta y_{2}$} ({2-normal(\dyb,0,0)}, 0+\dyb);
    \draw[color=red,->] (2,0) -- (2+\dxb*0.9,0+\dyb*0.9);

    \draw[color=red,domain=0.5:3.5] plot [smooth] (\x,{normal(\x,2,0)}) node[right] {};
    \fill[color=red] (2, 0) circle(0.08);
    \draw[color=red] (2+\dxb, 0+\dyb) circle(0.08);
    \draw[color=red,domain=-1.5:1.5,rotate=90] plot [smooth] (\x,{normal(\x,0,2)}) node[right] {};
\end{tikzpicture}

        \caption[Sketch how the Displacement Works]
        {
            Sketch how the displacement of the nodes works. The nodes
            get displaced by \(\Delta x\) and \(\Delta y\) drawn from the
            distributions displayed next to the points. The original
            square lattice is indicated by dashed lines.
        }
        \label{fig:displacement}
    \end{figure}\\
    This \(\sigma\) is also called \emph{disorder parameter} in the following.
    Because most sites will only have one nearest neighbor after the
    displacement, the lattice would collapse to many very small clusters.
    To avoid this, the new "nearest" neighbors are those sites connected
    by an edge. The edges are constructed according to
    one of the two in section \ref{ssec:graphtypes} defined rules,
    so that the lattice represents a proximity graph. Note that edges
    of a proximity graph do not cross each other, hence the 2D character
    of the lattice is preserved. The coupling constant \(J\) gets
    identified with edge weights. The weight of an edge \(E_{ij}\) is
    \(J_{ij} = \exp (\alpha (1-d_{ij}))\) where \(d_{ij}\) is the Euclidean
    distance between the nodes \(i\) and \(j\). The free parameter
    \(\alpha\) is set to \(\alpha = 0.5\) inspired by \cite{Lima2000}.
    The boundary is periodic e.g. nodes near the right edge can be
    connected to nodes near the left edge and vice versa. Analogous the
    top and bottom edges are connected. One can imagine that the model
    lives on the surface of a torus as pictured in fig. \ref{fig:torusRNG}.
    In subsequent graphics, the graphs will be unwrapped to rectangular
    shapes. Connections which cross a periodic boundary are indicated
    by edges which connect to an dashed node.
    \begin{figure}[htbp]
        \centering
        \includegraphics[width=0.45\textwidth]{images/torus}
        \caption[A Graph on a Torus to Visualise Periodic Boundary Conditions]
        {
            A graph on a torus to visualize periodic boundary conditions.
            Note that the lattice on this torus is not a square but has
            a height to width ratio of 1:4. At 1:1 the torus would cut
            itself. Hence, the torus represents the geometry of the model
            not perfectly, but gives very quick the right idea.
            %Also the shades are of course only a guide to the eye.
        }
        \label{fig:torusRNG}
    \end{figure}\\
    For \(\sigma = 0\) this is the standard Ising model with \(J = 1\),
    for which exists an analytic solution \cite{Onsager1944}. And for
    \(\sigma \gtrsim 1\) the nodes are distributed randomly. This case
    is already studied for constant \(J\) on the Delaunay triangulation
    in \cite{Janke1994}.\\

\subsection{Gabriel- and Relative Neighborhood Graph}
\label{ssec:graphtypes}
    A graph \(G(V,E)\) is a set of nodes \(V\) and edges \(E\).\\
    All here mentioned graph types are \emph{proximity graphs}. They are
    connecting nodes which are by some metric near to each other.
    Hence they are suited to generalize problems defined on regular
    lattices with nearest neighbor relationships, like the Ising model
    from the section \ref{ssec:isingmodel}.
    In this thesis the distance is always determined by the Euclidean
    metric in two dimensions, though in principle every metric in any
    dimension can be used.\\

    The Gabriel graph (GG) \cite{Gabriel1969} is a subgraph of the
    Delaunay triangulation. Two nodes \(i\) and \(j\) with distance
    \(d_{ij}\) are connected with an edge, if a circle with its
    center on half way between \(i\) and \(j\) and radius
    \(r = \frac d 2\) contains no other nodes. This area will be
    called \emph{lune} in the following. See also Figure
    \ref{fig:lunes}\subref{sfig:lunes:def}.\\
    The Relative Neighborhood graph (RNG) \cite{Toussaint1980} is a
    subgraph of the GG. Two nodes \(i\) and \(j\) with
    distance \(d_{ij}\) are connected, if no other node is in the
    \emph{lune}. The lune is defined as the intersection of two
    circles with radius \(r = d\) and centers on \(i\) and \(j\).
    See also Figure \ref{fig:lunes}\subref{sfig:lunes:def}.
    \begin{figure}[htbp]
        \centering
        \subfigure[Definition of the Lunes][]{
            \label{sfig:lunes:def}
            \documentclass{standalone}
\usepackage{tikz}
\usetikzlibrary{patterns}

\begin{document}
    \tikzset{
        hatch distance/.store in=\hatchdistance,
        hatch distance=10pt,
        hatch thickness/.store in=\hatchthickness,
        hatch thickness=2pt
    }

    \makeatletter
    \pgfdeclarepatternformonly[\hatchdistance,\hatchthickness]{flexible hatch no}
    {\pgfqpoint{0pt}{0pt}}
    {\pgfqpoint{\hatchdistance}{\hatchdistance}}
    {\pgfpoint{\hatchdistance-1pt}{\hatchdistance-1pt}}%
    {
        \pgfsetcolor{\tikz@pattern@color}
        \pgfsetlinewidth{\hatchthickness}
        \pgfpathmoveto{\pgfqpoint{0pt}{0pt}}
        \pgfpathlineto{\pgfqpoint{\hatchdistance}{\hatchdistance}}
        \pgfusepath{stroke}
    }
    \makeatletter
    \pgfdeclarepatternformonly[\hatchdistance,\hatchthickness]{flexible hatch nw}
    {\pgfqpoint{0pt}{0pt}}
    {\pgfqpoint{\hatchdistance}{\hatchdistance}}
    {\pgfpoint{\hatchdistance-1pt}{\hatchdistance-1pt}}%
    {
        \pgfsetcolor{\tikz@pattern@color}
        \pgfsetlinewidth{\hatchthickness}
        \pgfpathmoveto{\pgfqpoint{0pt}{\hatchdistance}}
        \pgfpathlineto{\pgfqpoint{\hatchdistance}{0pt}}
        \pgfusepath{stroke}
    }

    \begin{tikzpicture}
        \clip (-2,2.25) rectangle (2,-1.75);

        \begin{scope}
            \clip (-1, 0.5) circle(2.06155281281);
            %~ \fill[fill=blue!20] (1, 0) circle(2.06155281281);
            %~ \draw[pattern=north west lines] (1, 0) circle(2.06155281281);
            \draw[pattern=flexible hatch no,hatch distance=10pt,hatch thickness=0.7pt] (1, 0) circle(2.06155281281);
        \end{scope}

        %~ \fill[fill=white] (0, 0.25) circle(1.0307764064);
        %~ \draw[pattern=north east lines] (0, 0.25) circle(1.0307764064);
        \draw[pattern=flexible hatch nw,hatch distance=10pt,hatch thickness=0.7pt] (0, 0.25) circle(1.0307764064);
        \draw[thick] (0, 0.25) circle(1.0307764064);

        \draw[thick] (-1, 0.5) circle(2.06155281281);
        \fill (-1, 0.5) circle(0.1);
        \draw[thick] (1, 0) circle(2.06155281281);
        \fill (1, 0) circle(0.1);
        \draw[thick] (1, 0) -- (-1, 0.5);
    \end{tikzpicture}
\end{document}

        }
        \subfigure[RNG example][]{
            \label{sfig:lunes:rng}
            \includegraphics[width=0.3\textwidth]{images/RNG/L12S03.pdf}
        }
        \subfigure[GG example][]{
            \label{sfig:lunes:gg}
            \includegraphics[width=0.3\textwidth]{images/GG/L12S03.pdf}
        }
        \caption[Gabriel - and Relative Neighborhood Graph]
        {
            \subref{sfig:lunes:def} Lunes of RNG (hatched region) and
                GG (cross hatched region)
            \subref{sfig:lunes:rng} Example of a RNG on periodic
                boundary conditions. Periodic nodes are dashed.
            \subref{sfig:lunes:gg} Example of a GG on
                periodic boundary conditions. Periodic nodes are dashed.
        }
        \label{fig:lunes}
    \end{figure}\\
    To construct these graphs the simple way is to test for each
    pair of nodes if any other node lies in
    the lune of the pair. That is of complexity \(O (N^3)\), because
    there are \(N(N-1)\) pairs and for each \(N-2\) nodes to test. So
    the product is of order \(O(N^3)\)\\
    To reduce the complexity one can first create a Delaunay
    Triangulation in complexity \(O (N \log N)\)
    \cite{RNGCell} and test the criterion for each edge, because
    the Delaunay triangulation is a supergraph of both. But the
    implementation of a Delaunay triangulation algorithm is not trivial
    and the generation of the graphs is not time critical in the scope
    of this bachelor thesis.\\
    So a trade off is to use basically the simple method but only test
    the criterion for nodes which are near to the lune and abort if
    one node inside the lune is found. To determine which nodes are
    near the lune one can subdivide the area in \emph{cells} and save
    for each cell a list with nodes lying inside it like presented in
    \cite{RNGCell}.
    Now it is just necessary to test the nodes in the cells which
    resemble a rectangular bounding box of the lune. Most pairs will be
    far away from each other and the cells in the middle of the bounding
    box are completely inside the lune so that only one node has to be
    tested to discard an edge between them. Connected nodes are near to
    each other so that only very few cells have to be tested.\\
    Indeed this method reduced the time needed to construct a RNG with
    \(N=32^2\) and \(N=64^2\) by a factor of
    over \(15\) respectively \(40\). Though the complexity is still of
    order \(O(N^2)\) in the best case, because for every pair at least
    one check has to be performed.


    \section{Methods}
        \subsection{Thermodynamic Theory}
\label{ssec:theory}
    The disordered Ising model will be examined as a \emph{canonical system} in
    \emph{equilibrium}. A canonical system can exchange energy with a
    heat bath, thus has a constant temperature -- the temperature of the
    heat bath.
    Equilibrium is defined as a steady state, where
    the observables are only fluctuating but not changing in any
    particular direction. For example thermal equilibrium denotes the
    condition, that the system under scrutiny has reached the temperature
    of the heat bath. This is the case, once no energy is exchanged
    between them, thus the energy of the simulated system reaches a
    steady state.
    In a canonical ensemble the probability \(p_i\) of a state
    \(i\) is distributed according to a Boltzman distribution
    \begin{equation}
        p_i \propto e^{-\beta H_i}
    \end{equation}
    \begin{equation}
        \beta = \frac{1}{k_B T}
    \end{equation}
    Further the free energy \(F\) of a canonical ensemble is minimized
    in equilibrium and all observables can be derived from \(F\)
    in a straight forward way, as stated in every textbook about
    statistical physics (e.g.\ \cite{nolting2005}).\\
    Because of \(F=U-TS\) where \(U=\avg{H}\) is the internal energy and
    \(S\) entropy, one can guess, that for low \(T\) the internal energy
    will be low, and for big \(T\) the entropy will be high, to minimize
    \(F\). The Hamiltonian of the Ising model connects \(U\) and \(S\)
    by assigning low energies to states with high order. So \(U\) is low
    if all spins are aligned. In this case \(S\) as a measure of
    disorder is also low. Analogical the state of maximum \(U\) is also
    the state of maximum \(S\).
    This preliminary considerations make a phase transition at some
    \(T\), where the influence of the entropy on \(F\) becomes the same
    order of magnitude as the influence of the internal energy on \(F\),
    very plausible.\\
    To determine
    \begin{equation}
        F=-k_{B}T \ln{Z}
    \end{equation}
    or any expected value of an observable
    \begin{equation}
        \avg{O} = \frac{1}{Z} \sum_i O_i e^{-\beta H_i}
    \end{equation}
    one has to know the partition function
    \begin{equation}
        Z = \sum_i p_i = \sum_i e^{-\beta H_i}
        \label{eq:partitionFunction}
    \end{equation}
    where the sum goes over all possible states \(i\) of the system.
    Because every site can have two states, there are \(2^N\) different
    states of the system. For each the energy \(H_i\) has to be calculated
    to solve the sum from eq. \eqref{eq:partitionFunction} to gain \(Z\).
    Hence for small \(N\) the partition function is computable, but the system
    may show very different properties than in the thermodynamic limit.
    To minimize these finite size effects, it is desirable to examine
    systems with large \(N\). But \(2^N\) is a very rapidly increasing
    number, so there are next to infinity many different states, such
    that it is unfeasible to calculate the energy for each state, except
    for cases, where it is possible to solve it analytically.\\
    If not, one can get estimates of the observables for big \(N\) using
    Monte Carlo simulations, which are introduced in the next chapter.\\

    The observables which are measured in this thesis, are the mean
    magnetization per spin
    \begin{equation}
        m = \frac{1}{N} \sum_i s_i
    \end{equation}
    and the mean energy per spin
    \begin{equation}
        E = \frac{1}{N} \avg{H}
    \end{equation}
    As mentioned before, properties near the phase transition and the
    critical temperature \(T_c\), where the phase transition occurs, will
    be examined. The Ising system in two dimensions shows a second order
    phase transition, hence \(m\) and \(E\) are continous, but show at
    \(T_c\) an infinitly sharp slope i.e.\ the first derivative diverges.
    From statistical physics (compare \cite{nolting2005}) is known, that
    these derivatives can be expressed by fluctuations e.g.\ the specific
    heat can be expressed as
    \begin{equation}
        c = \frac{\partial \avg{H}}{\partial T} = k_B \beta^2 \avg{\brac{H-\avg{H}}^2}
    \end{equation}
    The specific heat is a measure for how much energy is needed to change
    the temperature of the system.
    Analogical the susceptibility
    \begin{equation}
        \chi = N \beta \avg{\brac{m-\avg{m}}^2}
    \end{equation}
    is a measure for how strong an outer magnetic field changes the
    magnetization of the system.
    Beside these observables from classical physics, a fifth observable
    the binder cumulant \cite{Binder1981}
    \begin{equation}
        g = \frac{3}{2}\brac{1-\frac{\avg{m^4}}{3\avg{m^2}^2}}
        \label{eq:binder}
    \end{equation}
    is considered.
    This is a dimensionless value, which can be used to determine the
    critical point. These five observables will be sufficient to analyse
    the phase transition in the scope of this bachelor thesis. All of them
    can be easily computed when \(m\) and \(E\) are measured. The next
    chapter will show, how to get estimates for them through Monte Carlo
    simulations.

\subsection{Monte Carlo Simulations}
\label{ssec:montecarlo}
    The idea behind Monte Carlo simulations is to take random samples of
    the observable, which should be measured, and estimate the mean of the observable from
    this samples. As an example the Monte Carlo integration chooses random
    points within the bounding box of the function and estimates the integral
    as the fraction of points below the function and points above multiplied
    by the area of the bounding box \cite{Katzgraber2011}.\\
    Analogical the technique can be applied to statistical ensembles,
    but instead of sampling points on an area, one samples states of the
    system.\\
    In statistical physics the expected value of an observable \(O\)
    is -- as also noted above -- calculated by
    \begin{equation}
        \avg{O} = \frac{1}{Z} \sum_i p_i O_i
    \end{equation}
    It is however possible, that there are few states that contribute
    massively more than others. Think of a canonical system at low \(T\),
    where \(p_i\) with low \(H_i\) contribute much more than high \(H_i\).
    But if one samples the \(2^N\) states evenly, it is probable to miss
    them, which possibly introduces big errors to the estimate of \(\avg{O}\).
    It is therefore desirable to sample only the states with high
    contributions to the sum.
    As noted before the system under scrutiny is canonical and therefore
    \(p_i\) is known. The states are distributed according to the Boltzman
    distribution. Hence \emph{Importance Sampling} can be utilized.\\
    Instead of sampling uniformly distributed random states, one samples
    states, which energies are distributed according to a Boltzman distribution.
    In fact this reduces the estimator to
    \begin{equation}
        O_M = \frac{1}{M} \sum_{i=0}^M O_i
    \end{equation}
    as shown in \cite{NewmanBarkema1999}. This is a very convenient form.\\
    But it is difficult to create a random state of a physical system
    e.g.\ the Ising system according to a given distribution. The simple
    method of creating uniformly distributed random states and reject
    them with probability \(p_i^{-1}\) depending on their energy is not
    efficient, because many generated states will be discarded and the
    computing time to generate them and caluculating their energy will
    be wasted.
    Hence one uses \emph{Markov Chains} to generate new states \(\nu\)
    from old ones \(\mu\). It is important that the transition probabilities
    \(A(\mu \to \nu)\) obey \emph{Detailed Balance} and \emph{Ergodicity}.
    \emph{Detailed Balance} means that the probability to leave a state is
    the same as the probability to enter the state in equilibrium
    \(p_\mu A(\mu \to \nu) = p_\nu A(\nu \to \mu)\) with \(p_\mu\) the
    probability to be in state \(\mu\). This ensures that the system can
    equilibrate and that the states are distributed according to the
    desired distribution in equilibrium \cite{NewmanBarkema1999}.
    And \emph{Ergodicity} requires that every possible state is reachable
    from every other state in finite time. \cite{NewmanBarkema1999} \cite{Katzgraber2011}
    Otherwise the samples might not be representative for the whole system.\\
    In this thesis three algorithms, which fulfill all requirements,
    were used. They will be shortly described in the following subsections.

    \subsubsection{Metropolis}
        A Metropolis Monte Carlo\cite{Metropolis1953} simulation of an
        Ising model will choose a random spin, calculate the energy change
        \(\Delta H\) defined in eq. \eqref{eq:dH} that would result
        from a flip of that spin and execute the flip with the probability \(A\)
        eq. \eqref{eq:metropolis} \cite{NewmanBarkema1999} \cite{Katzgraber2011}.
        \begin{equation}
            \Delta H = H(\nu) - H(\mu)\\
            \label{eq:dH}
        \end{equation}
        \begin{equation}
            A(\mu \to \nu) =
            \begin{cases}
                1                            & \Delta H \le 0 \\
                \exp{\brac{-\beta \Delta H}} & \Delta H > 0
            \end{cases}
            \label{eq:metropolis}
        \end{equation}
        So if a transition lowers the energy it is always done. This
        results in a high ratio between chosen spins and flipped spins.
        Therefore it minimizes the calculations needed for a change of
        the state. Also note that \(\Delta H\) is easy to calculate,
        because it is only affected by the spin of the neighbors of the
        chosen site.

    \subsubsection{Wolff}
    \label{sssec:wolff}
        Close to the critical temperature \(T_c\) the Metropolis
        gets slower. This is called \emph{critical slowing} down.\\
        Using a cluster algorithm like the Wolff
        algorithm \cite{Wolff1989} speeds things up.
        For an Ising model the Wolff algorithm builds a cluster of sites
        starting with a random site and adding neighboring sites of the
        same spin with probability \(P_{\mathrm{add}}\) from eq. \eqref{eq:wolffAdd}
        \begin{equation}
            P_{\mathrm{add}} = 1-\exp\brac{-2\beta J}
            \label{eq:wolffAdd}
        \end{equation}
        Where \(J\) is the coupling constant (explained in section
        \ref{ssec:isingmodel}). The neighboring sites of the added sites
        are also considered and so forth. When there are no more sites
        to add, the spin of every site in the cluster is flipped
        \cite[p. 91ff]{NewmanBarkema1999} \cite[p. 151f]{Katzgraber2011}.
        This leads fast to new uncorrelated states at the critical
        temperature because big clusters are flipped. But there are not
        much advantages at high or low temperatures. At low temperatures
        the cluster will consist of almost all sites such that all but
        very few spins will be flipped. At high temperatures the cluster
        will only contain very few sites.
        Both situations have no advantage against the Metropolis algorithm.\\
        So one would activate this algorithm near the critical temperature
        but use a simple Metropolis algorithm at high and low temperatures.

    \subsubsection{Parallel Tempering}
        Parallel Tempering\cite{ParallelTempering1986} simulates many identical systems at different
        temperatures and periodically swaps the spin configurations
        between two neighboring temperatures with probability \(P\) from
        eq. \eqref{eq:partemp} \cite[p. 169ff]{NewmanBarkema1999} \cite[S. 155ff]{Katzgraber2011}.
        \begin{equation}
            P((E_i,T_i) \to (E_{i+1},T_{i+1})) = \min\brac{1,\exp\brac{\brac{E_{i+1}-E_i}\brac{\frac{1}{T_{i+1}}-\frac{1}{T_i}}}}
            \label{eq:partemp}
        \end{equation}
        This has the advantages that correlation times of single
        temperatures are far smaller because their spin configuration
        gets often replaced by another uncorrelated configuration. In
        many cases the more important advantage is, that a system which
        is trapped in a local minimum at a given temperature, can travel
        to higher temperatures, leave its local minimum and cool down
        again in a lower minimum.\\
        In the case of a ferromagnetic Ising model the risk to get trapped
        in an local energy minimum is very low. Though the autocorrelation
        decreases significantly. In the scope of this thesis this comes
        for no cost, because one has to simulate for many temperatures
        to determine the critical temperature. The additional calculations
        to determine whether to swap configurations or not are small in
        comparison with those that would be needed to generate a new
        uncorrelated state without \emph{parallel tempering}.

    \subsubsection{Implementation Details}
        Here a mixture of the above three algorithms is used.
        Each sweep \(N\) Metropolis spin flips, one Wolff cluster flip
        and one parallel tempering swap are performed. Where \(N\) is the
        count of sites.\\
        Because it is not known before, where the critical temperatures
        \(T_c\) are located, the Wolff cluster algorithm is used for
        every temperature. The speed up at criticality is worth the
        moderate slow down at other temperatures.

    \subsubsection{Equilibration- and Autocorrelation Time}
    \label{sssec:eqtime}
        To generate states acoording to the Boltzman distribution at a
        given temperature \(T\), one starts with an arbitrary state
        and waits until it reaches thermal \emph{equilibrium}. Because
        equilibrium is defined as a steady state, one can determine if it
        is reached, by observing the change of the observables over the
        progressing simulation, as pictured in fig.\ \ref{fig:equiandauto}\subref{sfig:equiandauto:equiE}.
        The count of sweeps till
        equilibrium is called \emph{equilibration time} \(t_{eq}\).
        All measurements should start after this time.\\
        In fig.\ \ref{fig:equiandauto}\subref{sfig:equiandauto:equiE}
        the equilibrium is reached after approximately \(N_{s} \approx 50\) sweeps for
        either an initial condition of all spins up and all spins random. It
        does not harm to double that value to be save. Particularly because
        it is a random process, so that there can not be an exact value.
        \begin{figure}[htbp]
            \centering
            \subfigure[Example of an Equilibrating Ising System][]{
                    \label{sfig:equiandauto:equiE}
                    \includegraphics[width=0.47\textwidth]{plots/equiE}
            }
            \subfigure[Example of the Autocorrelation of an Ising System][]{
                    \label{sfig:equiandauto:autoM}
                    \includegraphics[width=0.47\textwidth]{plots/autoM}
            }
            \caption[Examples for Equilibration and Autocorrelation]
            {
                \subref{sfig:equiandauto:equiE} Example of a Ising system
                    \(L=64\) reaching equilibrium at \(T=2.36\) and
                \subref{sfig:equiandauto:autoM} the autocorrelation of an
                    Ising system \(L=64\) at \(T=2.40\) (only Metropolis
                    sweeps -- otherwise the decline is too steep to show)
                    on half logarithmic axis.
                    The straight line is an exponential fit \(\exp(-t/\tau)\)
                    with \(\tau = 342(1)\).
            }
            \label{fig:equiandauto}
        \end{figure}\\
        Because every state is generated from the state before, measurements
        of subsequent states are correlated. To determine when two states
        are independent, one calculates the normalized autocorrelation function
        \(\frac{\chi(t)}{\chi(0)}\) with
        \begin{equation}
            \chi(t) = \int \mathrm{d} t' \, [m(t') -\avg{m}][m(t'+t)-\avg{m}]
        \end{equation}
        which should decay exponentially
        \(\chi(t) \propto \exp(t/\tau)\). This is visible in the half
        logarithmic plot \ref{fig:equiandauto}\subref{sfig:equiandauto:autoM}.
        To get the autocorrelation time one can either fit a exponential
        function \(\exp(-t/\tau)\) like in fig.\ \ref{fig:equiandauto}\subref{sfig:equiandauto:autoM}
        or integrate \(\tau = \int \frac{\chi(t)}{\chi(0)} \de t\).
        \(\tau\) is an estimate after which time two samples are not
        correlated anymore. \cite[p. 59ff]{NewmanBarkema1999} \cite[p. 150f]{Katzgraber2011}.
        To make sure that the error is not underestimated one should wait
        \(2\tau\) sweeps between two measurements.
        The autocorrelation time is of course dependent on the temperature.
        For example for the standard Metropolis algorithm the fluctuations
        are strong at high temperatures and subsequent
        states are more dissimilar and therefore less correlated than at low
        temperatures, where less spins are flipping. But the longest
        autocorrelation times are encountered at the critical temperature.
        This effect is called \emph{critical slowing down} and is
        characterized by the \emph{dynamical critical exponent} \(z\)
        \cite{SwendsenWang1987}. The dependence of the autocorrelation time
        \(\tau\) on the system size \(L\) is at \(T_{c}\) given by \(\tau \propto L^z\).
        More general the power law \(\tau \propto \xi^z\) holds true, where
        \(\xi\) is the \emph{correlation length}, which diverges at
        \(T_{c}\) and is then limited by the size of the simulated lattice.
        As mentioned in section \ref{sssec:wolff}, the Wolff cluster algorithm
        decreases \(z\) dramatically. This causes the course of the autocorrelation
        time in dependence on temperature to change significantly as shown in
        section \ref{ssec:results:autocorr}. Also according to \cite{NewmanBarkema1999}
        \(z\) is independent of the lattice structure, which ensures that
        the simulation will benefit from the Wolff cluster algorithm at
        any \(\sigma\).


    \section{Results}
        \subsection{Critical Temperature}
    The evaluation of the Binder cumulant's intersections, yields the
    critical temperatures \(T_c\), which are normalized and plotted in
    fig. \ref{fig:Tc}.
    \begin{figure}[htbp]
        \centering
        \subfigure[][]
        {
            \label{sfig:Tc:RNG}
            \includegraphics[width=0.45\textwidth]{plots/RNG_Tc_norm}
        }
        \subfigure[][]
        {
            \label{sfig:Tc:GG}
            \includegraphics[width=0.45\textwidth]{plots/GG_Tc_norm}
        }
        \caption[Critical Temperature over different disturbance parameters]
                {Normalized critical temperatures over different
                 disturbance parameters for
                 \subref{sfig:Tc:RNG} the Relative Neighborhood Graph and
                 \subref{sfig:Tc:GG} the Gabriel Graph.
                }
        \label{fig:Tc}
    \end{figure}
    % I don't know what all that means...

\subsection{Critical Exponents}
    For \(\sigma \in \{0,0.1,0.5\}\) a finite size scaling analysis was
    performed to determine the critical exponents \(\beta, \gamma, \nu\)
    using \texttt{autoscale.py} \cite{autoscale2009}. The values for
    \(\sigma = 0\) are analytically known \cite{Pelissetto2002}. The
    values for all other \(\sigma\) should be the same according to ???
    [citiation needed]. Like in tab. \ref{tab:critExp} to see, most values
    are matching the expectations. Most \(\beta\) seem to be a bit too
    big, but they are close enough to the expectations to be explained
    by the fact that small systems (\(L=32,64\)) were used for the
    analysis. [citation needed]
    \begin{table}[htbp]
        \center
        \begin{tabular}{l l l l l}
            \toprule
             & \multicolumn{1}{c}{\(\sigma\)} & \multicolumn{1}{c}{\(\nu\)} & \multicolumn{1}{c}{\(\gamma\)} & \multicolumn{1}{c}{\(\beta\)}\\
            \midrule
            exact (\cite[p. 59]{Pelissetto2002}) & \multicolumn{1}{c}{\(0\)} & \multicolumn{1}{c}{\(1\)} & \multicolumn{1}{c}{\(-\frac{7}{4}\)} & \multicolumn{1}{c}{\(\frac{1}{8}\)}\\
            \midrule
            Gabriel      & 0.0 & 1.008(4) & -1.735(2) & 0.1262(4)\\
                         & 0.1 & 1.02(1)  & -1.744(5) & 0.133(6) \\
                         & 0.5 & 1.009(8) & -1.75(1)  & 0.125(13)\\
            \midrule
            Relative N.  & 0.0 & 1.007(2) & -1.739(2) & 0.130(1) \\
                         & 0.1 & 0.99(1)  & -1.746(5) & 0.133(4) \\
                         & 0.5 & 1.00(2)  & -1.75(2)  & 0.143(13)\\
            \bottomrule
        \end{tabular}
        \caption{Critical exponents for different \(\sigma\)}
        \label{tab:critExp}
    \end{table}

    %~ <J>\(\sigma\), Anmerkung zur Anomalie des GG und der Ähnlichkeit von <J> zu \(T_c\)\\
%~
    %~ Erklärung zu Autokorreltationszeit \(\tau\) und Equilibrierungszeit \(t_eq\)\\
    %~ Darstellung von Suszeptibilität \(\chi\), spezifischer Wärme \(c\), mittlerer Magnetisierung \(<m>\) über Unordnungsparamter \(\sigma\)\\
    %~ Bestimmung der kritischen Punkte\\
        %~ Polyfit 4ten Grades durch Binder Kumulante \cite{Binder1981}\\
        %~ Finite Size Scaling Vergleich der Exponenten (AutoScale \cite{Melchert2009}, Vergleich \cite[S. 59]{Pelissetto2002})\\



    \section{Conclusion}
        In this thesis the properties of the disordered Ising ferromagnet on two
proximity graphs was studied. This model is within the universality of
the standard Ising ferromagnet. Its critical temperature decreases with
increasing disorder on a RNG. On a GG it first jumps, increases up to
a maximum and decreases after that. The course of the critical
temperatures can be made plausible observing the degree of the underlying
graph, but no explicit formular was found.

\subsection{Outlook}
    The behavior of the normalized critical temperatures in dependence
    on \(\sigma\) of the Relative Neighborhood graph and Gabriel graph
    are qualitative similar to each other. One could study the behavior
    of other proximity graphs like the minimum spanning tree or the
    Delaunay triangulation, to examine if it is also similar. Further
    one could examine other graph ensembles to determine, if this is a
    more universal property.\\
    Also it would be interesting to compare this results with a the
    results obtained by the same experiments with \(\alpha = 0\), which
    implies that the coupling constant is independent from the distance
    on the sites. That way the critical temperature would be more
    straight forward normalized by the degree of the graph. This way it
    could be determined if the qualitative similarities of the
    normalized temperatures over the disorder parameter of different
    graph types are related to the choice of the function determining the
    coupling constant.


    \clearpage
    \pagestyle{empty}
    The simulations were performed at the HPC Cluster HERO, located at
the University of Oldenburg (Germany) and funded by the DFG through
its Major Research Instrumentation Programme (INST 184/108-1 FUGG)
and the Ministry of Science and Culture (MWK) of the Lower Saxony
State.
The authors would also like to thank O. Melchert and T. Dewenter
for the fruitful discussions.


    \clearpage
    \bibliography{lit}
    \bibliographystyle{alpha}
    %~ \bibliographystyle{amsplain}
\end{document}
